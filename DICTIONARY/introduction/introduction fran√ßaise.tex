% 3 façons de citer: \cite, \textcite, \parencite

\languefra

\chapter*{Introduction}

La présente introduction est divisée en cinq volets. Elle s'ouvre sur une présentation de la langue (§\ref{sec:lang}), du dialecte étudié et des participant·es au travail de documentation (§\ref{sec:dialect}). Les objectifs et la chronologie de l'enquête de terrain et du travail lexicographique sont ensuite exposés (§\ref{sec:chronologie}). Le §\ref{sec:guide} constitue un Guide d'utilisation. Enfin sont esquissées quelques perspectives de recherche (§\ref{sec:recherche}), qui visent à expliciter certains enjeux du travail lexicographique au sujet de langues rares.

% Ensuite une présentation du processus au fil duquel s'est élaboré (et continue de s'élaborer) le présent ouvrage \label{sec:plan}: un historique de la collaboration entamée en 2006 avec Mme Latami Dashilamu, locutrice de référence et co-autrice du présent travail. Puis quelques réflexions

\section{Les Na et la langue na (mosuo)}
\label{sec:lang}

Les Na (ou Mosuo) sont célèbres, tant auprès des ethnologues et anthropologues que d’un public plus large, pour leurs structures familiales singulières. Cai Hua les décrit comme \emph{une société sans père ni mari} \parencite{cai1997}, mais un de ses lecteurs \parencite[147]{wellens2003} relève que l'auteur écarte discrètement, dans sa démonstration, les observations qui ne s'accordent pas avec la généralisation résumée par le titre de son ouvrage. Cai Hua force le trait afin de camper le portrait d'une société diamétralement opposée à la société chinoise (Han), au sein d'un système abstrait d'oppositions binaires \parencite{cai2008}. Une description nettement plus adéquate consiste à décrire les Na comme un cas exemplaire de société à la fois matrilinéale et matrilocale. Une telle caractérisation est présente dès les premiers travaux ethnologiques chinois \parencite{zhanetal1980,yanetal1983,weng1993,shih1993,shih2010}; cependant, ces études se focalisent sur les théories du lignage, tout en décrivant une diversité de pratiques qui contraste fortement avec l'interprétation rigide qui en est donnée.

L'ethnographie approfondie de villages na \parencite{milan_na_2016,milan_tourisme_2019} invite à interpréter les structures familiales des Na dans le contexte spécifique des marches sino-tibétaines, région marquée par une diversité exceptionnelle de structures familiales.
Au lieu de s’attacher à une classification fondée sur des catégories telles que la «matrilinéalité», il est plus pertinent d’interpréter ces diverses structures en termes de dynamique des \emph{maisons}. Cette perspective, fruit des toutes premières critiques du paradigme lignagier, mérite d'être intégrée comme un élément central dans toute analyse des structures familiales des Na \parencite{zhou_zigong_2003,hosana_2006}. Elle met en effet en lumière une cohérence que les théories du lignage peinaient à saisir: \emph{formes de cohabitation et pratiques matrimoniales constituent des stratégies qui visent à assurer la reproduction sociale des maisons}. Ainsi, «prendre femme» \emph{(chumi shei)} constitue, pour une maison, un moyen d’assurer son équilibre et sa pérennité\footnote{Ce thème est abordé dans une communication de Pascale-Marie Milan, «Gender balance strategies and social engineering pressures: “Taking wives” (\emph{chumi shei}) in the context of the Na family of Lijiazui in feudal times and in present-day China», exposée lors de l'Atelier «Family resemblances: kinship practices along the Sino-Tibetan borderlands», CNRS-École Française d'Extrême-Orient, octobre 2024.}. Ces pratiques permettent aux maisons de s'adapter aux différents régimes politiques qui se sont succédé, cela depuis des époques historiques nettement antérieures à la prise de pouvoir communiste en 1958. Dans cette perspective, l'analyse des dynamiques d'adaptation des familles na au changement social qu'amène le développement rapide du tourisme dans la région du lac Lugu ces dernières années \parencite{walsh_living_2001,mattison2010,milan_tourisme_2019} constitue un prolongement cohérent des recherches anthropologiques menées sur les structures familiales chez les Na. Ces travaux contribuent, décennie après décennie, à renouveler en profondeur les approches de l'anthropologie de la parenté et de la famille.

En comparaison des études ethnographiques et anthropologiques, les études linguistiques au sujet des Na sont relativement peu développées. La langue na (mosuo) n'a guère fait l'objet d'études approfondies jusqu'aux premières années du XXI\textsuperscript{e} siècle. Cette langue présente pourtant des caractéristiques tout à fait fascinantes, dont une morpho-tonologie très riche: les tons jouent un rôle dans l'ensemble du système grammatical \parencite{michaud2017}. Le présent dictionnaire s'inscrit dans le cadre de l'entreprise plus générale qui vise à produire une documentation fiable et abondante au sujet de la langue na de Yongning.

L'autonyme (endonyme) de la langue, \phonologie{nɑ˩-ʐwɤ˥}, s'écrit ‘Narua’ dans l'orthographe romanisée proposée par \textcite{dobbs_ortho_2018} et utilisée dans le présent dictionnaire. Cette langue est aussi connue sous le nom de «Mosuo», romanisation du chinois \pcmn{摩梭}\footnote{Ce terme est parfois noté «Moso», notamment par Chuan-Kang \cite[][]{shih1993,shih2010}.}. Son code dans le registre de langues «Ethnologue»  \parencite{lewisetal2016} est \textsc{nru},
et son code Glottolog est \textsc{yong1270} \parencite{Nordhoff2012}. %La langue est présentée dans une monographie au sujet du système tonal de ce parler \parencite[2-8]{michaud2017}.
Le nom «na de Yongning» (en anglais: «Yongning Na») a été créé par Liberty \textcite{lidz2006} en associant à l'endonyme du peuple concerné le nom du lieu où la langue est parlée: la plaine de Yongning \pcmn{永宁}, un~bassin situé dans le sud-ouest de la Chine, à la frontière entre le Yunnan et le Sichuan, à une latitude de 27°50' N et une longitude de 100°41' (voir la carte \ref{map:1-1}). Yongning est proche du lac Lugu \pcmn{泸沽湖} (en na: \emph{Loshu xienami} \phonologie{lo˧ʂv̩˩-hi˩nɑ˧mi˧}), un lac d'une cinquantaine de kilomètres carrés qui crée un microclimat propice à l'agriculture malgré l'altitude élevée (environ 2~800 mètres au-dessus du niveau de la mer).

\begin{figure}
	\centering
	\caption{Carte de la plaine de Yongning et des environs. \emph{Créée par Jérôme Picard. Sources: Geofabrik, ASTER GDEM (une réalisation METI et NASA) et OpenStreetMap.}}
	\includegraphics[width=.68\textwidth]{Lien vers images/PDF_CMYK_1point4.pdf}
	\label{map:1-1}
\end{figure}

Le nombre total de locuteurs a été estimé à environ 40~000 sur la base d'enquêtes datant de la fin des années 1950 \parencite[107]{heetal1985}; le même chiffre est repris par \textcite{yang2009}. Le nombre de locuteurs compétents dans les années 2020 est nettement moins élevé: un remplacement de la langue par le mandarin est en cours.

La région de Yongning est devenue, depuis les années 1990 et surtout au XXI\textsuperscript{e} siècle, une destination hautement touristique. L'étude des conséquences anthropologiques de cette nouvelle industrie \parencite{milan_tourisme_2019} fait écho à l'intuition d'un de nos poètes:

\begin{quote}
    « [L]e tourisme s’est répandu comme une eau en crue et qui charrie des épaves. Il disloque le plus lointain des villages, il claironne dans la chambre la plus secrète des temples ses formules creuses et arrogantes, stéréotypes qui sont bien de notre Occident, hélas, dont la brochure publicitaire, cette parole de personne à propos de rien, est une des inventions spécifiques (…) [L]e texte touristique peut contraindre les habitants de ces lieux à se voir de par le dehors d'eux-mêmes, à se raconter comme le désire la paresse des arrivants, à substituer à leurs intuitions et leurs souvenirs les dessins bariolés qu'il faut à des visiteurs décontenancés ou hilares. » (Yves Bonnefoy, \emph{L'Inachevable. Entretiens sur la poésie}, 1990-2010. Paris: Albin Michel, 2014, p. 232.)
\end{quote}

Ce contexte prête un sens particulier au travail qui consiste à se mettre à l’écoute de la langue des Na, à l’heure de son remplacement rapide par le chinois mandarin (langue de l’école, des média, et du tourisme). Il y a lieu d'espérer que les enregistrements réalisés sur le terrain, et progressivement transcrits, traduits et annotés, parviennent à conserver et restituer, dans sa fraîcheur, la parole que notre collaboratrice (co-autrice du présent volume) a bien voulu livrer au public et confier à la postérité.

Le dictionnaire se veut un outil d'accès à ces documents, témoignages d’une culture qui s’estompe: celle des Na, cultivateurs de la plaine de Yongning. Le travail de documentation dans lequel s'inscrit le présent travail repose sur la conviction selon laquelle la langue et la culture na présentent un grand intérêt par elles-mêmes, pas uniquement en tant qu'elles appellent prise en compte dans des études typologiques ou historiques, en tant que \emph{points de données} permettant de sustenter des modèles statistiques ou des activités théorisantes.

	\begin{quote}
    Ce que nous tenons pour un monde (…) n’est-il pas tout d’abord, jusqu’au lointain des espèces, le champ qu’ont défriché et labouré des hommes occupés essentiellement de leur présence à un lieu, leurs objets reflétant donc ce désir, leurs savoirs ne cherchant en correspondances diverses que l’écho de ce projet de fonder ? Dans ce cas les mots ne seraient que l’inscription, embroussaillée aujourd'hui sinon effacée, d’une pensée de la présence, élaborée comme telle, dans la « pierre » de ce dehors (…). Si les mots, disons cela autrement, n’étaient pas pour atteindre à l’en-soi des choses mais pour approfondir le rapport d’une présence humaine et d'un lieu ? S’ils étaient moins le « dictionnaire » de ce qu’on dira la nature que les marches de ce pays – ce « séjour », disait Mallarmé – où, par la grâce d'un horizon que des noms resserrent, nous pouvons être ? Alors, bien sûr, chaque rencontre d'un être serait notre destin qui se joue, non la science qu’on vérifie. Et notre conscience en éveil se devrait de l’interroger comme son possible, même si déjà le temps nous en prive, au lieu de l’effacer de notre parole comme un exemple de plus de quelque schème éternel. (Yves Bonnefoy, « La fleur double, la sente étroite: la nuée », réimpr. in \emph{La Vérité de parole et autres essais}, Paris: Gallimard, 1995, pp. 566-567.)
\end{quote}

\subsection*{Indications bibliographiques}

Parmi les références importantes au sujet de la langue na, on citera une grammaire de référence d'un dialecte proche de celui décrit ici: celui de Luoshui \pcmn{落水} \parencite{lidz2010}. Les systèmes de tons de dialectes jusqu'alors non décrits ont fait l'objet d'analyses détaillées \parencite{a2016,dobbsetal2016,michaud2017,fily_documentation_2022}. En ce qui concerne les travaux lexicographiques, \emph{An anthology of everyday words and expressions in the Mosuo language} \parencite{zhibaetal2013} présente du vocabulaire et des expressions classés par champ sémantique. Les auteurs sont un locuteur natif de la région du lac Lugu (\pcmn{泸沽湖}) et un docteur en linguistique de l'université du Yunnan. Leur travail vise à couvrir la plaine de Yongning et la région du lac, mais c'est, en pratique, sur la plaine de Yongning que se concentre l'essentiel de leur travail (p.~2). Les imprécisions dans la notation phonétique sont si nombreuses qu'elles rendent le volume peu fiable en tant qu'ouvrage de référence. Les oppositions de timbre vocalique ont manifestement constitué un défi pour le linguiste de l'équipe, dont la formation était principalement axée sur la théorie et la pratique de l'enseignement du chinois langue étrangère. Ainsi, le nom de la montagne \phonologie{kɤ˧mv̩˧˥} est transcrit en \phonétique{gə⁵⁵mu⁵⁵}, avec une initiale voisée (p. 17 et ailleurs). Le nom donné à cette montagne en chinois, \pcmn{格姆山}, \emph{Gemu} en chinois romanisé, pourrait bien avoir contribué à induire les auteurs en erreur. Inversement, l'adjectif \phonologie{dʑɤ˩} `bon' est transcrit comme \phonétique{tɕɑ¹³}, avec une initiale non voisée. Certains phonèmes, tels que les consonnes uvulaires, sont absents des notations. Il reste donc clairement une lacune à combler dans les travaux lexicographiques sur le na. Les dictionnaires du naxi, langue géographiquement voisine et prochement apparentée, fournissent un brillant exemple à imiter \parencite{heetal2011,pinsonetal2012}.

\section{Dialecte et participant·es au travail de documentation}
\label{sec:dialect}

\subsection{Consultant·es linguistiques}

Sauf indication contraire, les données sont fournies par le second auteur, Mme Latami Daeshi Lhamu (\phonologie{lɑ˧tʰɑ˧mi˥ ʈæ˧ʂɯ˧-lɑ˩mv̩˩}; en chinois: \pcmn{拉他咪打史拉姆} Lātāmǐ Dǎshǐlāmǔ). Elle est née en 1950 dans le hameau de Alawua \phonologie{ə˧lɑ˧-ʁwɤ\#˥}, près du monastère de Yongning. Les coordonnées administratives de ce hameau sont les suivantes: province du Yúnnán, municipalité de Lìjiāng, comté autonome Yí de Nínglàng, bourg de Yǒngníng, village de Ālāwǎ (\pcmn{云南省丽江市宁蒗彝族自治县永宁镇}\footnote{Le \emph{canton} (\pcmn{乡}) de Yongning a été recatégorisé en \emph{bourg} (\pcmn{镇}) le 11 février 2019.}\pcmn{阿拉瓦村}). Le parler na de Alawua correspond au code Glottolog \textsc{yong1288}, tandis que le parler de variété Lataddi (\emph{Lataddi Narua}), parlé sur les rives du lac Lugu, a reçu le code \textsc{lata1234}.

Le choix de mener l'enquête en un seul et unique lieu, et de travailler pour l'essentiel avec une seule et même personne, tient au thème central des premières enquêtes: le système tonal. Les dialectes de l'aire linguistique na sont en effet nettement différenciés au plan tonal (bien plus que dans l'aire linguistique naxi). Les systèmes tonals des différents villages sont notablement différents, et la diversité géographique se double d'importantes différences entre groupes sociaux et entre générations. La chose à faire paraissait, à l'évidence, consister en une description et une analyse approfondies de la langue telle qu'elle est parlée par une personne, sans exclure, bien sûr, quelques incursions dans d'autres idiolectes et dialectes en fonction des occasions qui se présenteraient. La collaboration s'est bien engagée, à tel point que Mme Latami a accepté (en 2024) la proposition de figurer en tant que coautrice du présent volume.

Les données provenant d'autres locutrices et locuteurs sont indiquées dans le dictionnaire au moyen de codes à deux lettres, fournis dans la colonne la plus à gauche du tableau \ref{table:ConsultantsTable}. Le tableau indique la correspondance avec les codes locuteur dans la base de données d'A.\ Michaud (par exemple `F4' pour la consultante principaleet coautrice de ce volume). La numérotation des consultants na en fonction de ces codes locuteur est discontinue (F4, F5, F6, F22 et M18, M21, M23, plutôt que F1 à F4 et M1 à M3) du fait que ces codes ont été attribués de façon chronologique aux consultant·es d'enquêtes au sujet des trois langues du groupe naish: naxi, na et lazé.

\begin{longtblr}[
  caption = {Consultant·es linguistiques},
  label = {table:ConsultantsTable}
]{
  colspec = {X[0.6,l,m]X[1.1,l,m]X[l,m]X[l,m]X[0.6,l,m]X[0.3,l,m]},
  rowhead = 1
}
  \hline
  {initiales} & {nom en orthographe} & {nom en alphabet phonétique} & {nom en chinois} & {année de naissance} & {code locuteur} \\
  \hline
        La & Latami Daeshi Lhamu & \phonologiebis{lɑ˧tʰɑ˧mi˥ ʈæ˧ʂɯ˧-lɑ˩mv̩˩} & 拉他咪打史拉姆 & 1950 & F4 \\
        Gi & Gisso & \phonologiebis{ki˧zo˧} & 郭给若 & 1973 & F5 \\
        Qi & Qiddeu & \phonologiebis{tɕʰi˧ɖv̩\#˥} & 郭园芝 & 1987 & F6 \\
        Sg & Siggeema & \phonologiebis{sɯ˧gɯ˧mɑ˧} & 思格玛 & 1987 & F22 \\
        Da & Latami Daeshi Daedeu & \phonologiebis{lɑ˧tʰɑ˧mi˥ ʈæ˧ʂɯ˧-ʈæ˩ʈv̩˩} & 拉他咪王勇 & 1972 & M18 \\
        Jj & Ho Jjacee & \phonologiebis{ho˧dʑɤ˧tsʰe˥} & 何甲泽 & 1942 & M21 \\
        Dd & Ddeezzhi  & \phonologiebis{ɖɯ˩ɖʐɯ˧} & 何独知 & 1974 & M23 \\
  \hline
\end{longtblr}

Le petit cercle des collaboratrices et collaborateurs a connu une croissance organique. Le premier contact était Latami Daeshi (M18), qui a entrepris de rechercher des consultants linguistiques du village. Comme expliqué dans la monographie de 2017 \parencite[28-29]{michaud2017}, cette recherche ne s'est pas avérée fructueuse dans un premier temps, et c'est sa mère (F4) qui a accepté d'endosser le rôle de consultante-enseignante. Elle demeure une référence constante, source principale dans le travail du linguiste, partenaire de dialogue quotidien sur le terrain, et co-autrice du travail. Sa belle-fille (F5) a accepté de prêter main-forte, ainsi qu'une nièce de cette dernière (F6). Plus tard, un cousin de F4 (M21) et son fils (M23) ont à leur tour participé à l'enquête. Enfin, Siggeema (F22) est originaire du village de Wualabbi; la contribution qu'elle apporte consiste à participer au travail de test et de correction de l'orthographe avec Roselle Dobbs, et de fournir des informations de première main sur divers sujets de phonétique, de sémantique, de grammaire et de dialectologie na.

Le petit échantillon qui s'est ainsi constitué au fil du temps, sans plan d'ensemble arrêté, n'est pas équilibré selon tel ou tel critère, et ne se prétend pas représentatif. On peut néanmoins relever rétrospectivement certaines de ses caractéristiques.

En termes de générations, l'échantillon présente une certaine diversité. Les locutrices et locuteurs appartiennent à trois générations successives. F4, co-autrice du volume, née en 1950, est de la même génération que M21. La génération suivante est représentée par M18 (son fils), F5 (sa belle-fille) et M23 (fils de M21). F6 et F22 sont plus jeunes, au point de pouvoir être considérées comme appartenant à une troisième génération.

En termes de savoirs:

\begin{itemize}
    \item F4 est la plus immergée dans la culture na, ayant résidé de façon continue à Yongning pendant soixante ans (de sa naissance jusqu'en 2010).
    \item F5 et M23 appartiennent à la génération suivante de locutrices et locuteurs qui sont resté·es au village (à date de 2024), pour laquelle le chinois local (mandarin du sud-ouest) est une langue parlée couramment dès le jeune âge, proche d'une \emph{seconde langue maternelle}.
    \item M18 a une adhésion de cœur à la langue et la culture na, et une pratique quotidienne de la langue avec sa mère (F4). Il s'exprime dans cette langue avec aisance\footnote{On peut consulter en ligne huit enregistrements (audio et vidéo) qu'il a réalisés sur les thèmes suivants:
    \href{https://doi.org/10.24397/pangloss-0007734}{1: Yongning et les Na}; \href{https://doi.org/10.24397/pangloss-0007740}{2: l'ethnonyme ‘Mosuo’: son histoire et sa pertinence}; \href{https://doi.org/10.24397/pangloss-0007730}{3: récit personnel}; \href{https://doi.org/10.24397/pangloss-0007736}{4} et \href{https://doi.org/10.24397/pangloss-0007738}{5}: retour sur la collaboration avec le linguiste-enquêteur; \href{https://doi.org/10.24397/pangloss-0007742}{6: passage en revue des études na (mosuo)}; \href{https://doi.org/10.24397/pangloss-0007728}{7: les valeurs culturelles dont héritent les jeunes}; \href{https://doi.org/10.24397/pangloss-0007732}{8: la symbolique du foyer dans la culture na (mosuo)}.}.
    Pour autant, le titre \emph{Le Chagrin et la douleur} (\pcmn{心碎与忧伤}) qu'il a donné à un recueil de ses travaux \parencite{latami2016} laisse deviner ce qu'il y a de contrariant à vivre sa vie de lecteur, de chercheur, d'auteur dans une langue (le chinois) qui n'est pas celle de la culture maternelle, aimée et étudiée.
    \item M21 a habité dans diverses localités de la région, ce qui a naturellement amené à des ajustements dialectaux au fil de ses affectations successives. Au final, ce processus a quelque peu éloigné son parler de celui de la locutrice de référence. Nous lui sommes particulièrement reconnaissants des efforts menés pour participer à l'enquête malgré sa surdité partielle.
\end{itemize}
%Les personnes ayant participé à l'enquête sont divisées ci-dessous en deux groupes, selon qu'ils sont locuteurs du dialecte de Alawua en tant que ``mosuo langue première", ou extérieurs au groupe, intéressé·es par la langue et la culture na (et familiers de ce dialecte à des degrés divers: locuteurs  d'autres dialectes, ou simplement de ``mosuo langue étrangère" découvert à l'âge adulte). Cette séparation en deux groupes a bien sûr quelque chose d'arbitraire:

Aujourd'hui comme hier, un espoir constant est d'associer étroitement les locutrices et locuteurs aux recherches menées au sujet de leur langue \textcite{bouquiauxetal1971}. Le choix d'inscrire le nom de la locutrice aux côtés de celui du linguiste, comme co-autrice du présent ouvrage, s'inscrit dans cette logique. Son fils Latami Wangyong n'a pas attendu les échanges avec Alexis Michaud pour entrer dans le cercle des savants. Ethnologue spécialiste de la culture na, il apporte au présent ouvrage une contribution qui tient à son regard de savant, plus qu'à un statut de locuteur natif qui serait la garantie d'une science infuse en matière de jugements linguistiques.

\subsection{Autres collaboratrices et collaborateurs}

Parmi les «Autres collaboratrices et collaborateurs» figurent:

\begin{itemize}
    \item Pascale-Marie Milan, qui au fil des relectures apporte au travail son regard expert d'ethnologue-anthropologue. À partir de la version 2, sa contribution tout à fait essentielle est reconnue par le statut de co-autrice.
    \item Roselle Dobbs (Ddeema Lhaco, \pcmn{杜玫瑰}) et ses amis et collaborateurs na, à qui je dois, outre l'orthographe dont Roselle a doté ce dictionnaire, un important flux d’informations, de corrections et de conseils depuis les débuts de mes recherches au sujet du na de Yongning. À partir de la version 2, ce travail est mentionné en page de titre.
    \item Benjamin Galliot, qui assure depuis 2016 la mise en forme du dictionnaire avec un degré exceptionnel d'ambition logicielle et typographique, doublé d'une grande précision à chaque étape de son travail. À partir de la version 2, son rôle est reconnu en page de titre.
\end{itemize}

Le terme de \emph{consultants}, par lequel sont désignés (par un calque du terme anglais \emph{language consultants}) ceux qu'on désignait autrefois comme \emph{informateurs}, pourrait au fond également être employé pour ces collaboratrices et collaborateurs, qui fournissent des conseils experts, des corrections et des observations de tous ordres au fil du travail. Cet usage (assurément trop paradoxal au vu des pratiques actuelles pour qu'on soit tenté de l'adopter) aurait pour avantage de ne pas poser de frontière entre deux groupes: les locuteurs natifs et les autres, mais au contraire de souligner que le dictionnaire est le fruit d'un travail d'équipe.

%Une telle division, qui tend à essentialiser les Na comme héritiers légitimes des savoirs par transmission ``verticale'', peut aller de pair avec un cloisonnement insidieux qui tendrait à les cantonner dans un rôle de pourvoyeurs de savoirs qu'exploiteraient ensuite des personnes extérieures au groupe. Or


\subsection{Le parler de Alawua et son contexte dialectal}

Les contours de la \emph{communauté na} au sein de laquelle est parlé le dialecte de Alawua ne présentent pas nécessairement le même degré de netteté que dans les sociétés nord-américaines ou australiennes qui constituent une référence importante dans les travaux méthodologiques contemporains\footnote{Ainsi, le terme de «communauté» apparaît à trente et une reprises dans un article au sujet des méthodes d'enquête linguistique en immersion écrit par un collègue affilié à l'Université de Sidney: ``Enhancing data collection through linguistic competence in a field language: Perspectives from rural China''.  L'article de Manuel David González Pérez repose sur des réflexions approfondies au sujet d'enquêtes de terrain sur une langue yi du Yunnan; les références citées, au nombre de près de 200, portent sur des langues très diverses, mais sont toutes anglophones (exceptées deux en allemand et deux en espagnol), ce qui comporte un risque de biais culturel. Comme le relèvent des anthropologues, ``the term “community” (...) does not refer to the same idea in (...) different countries. In practice, it can be translated in very different ways, depending on what it refers to within state policies'' \parencite{dallesmarechal_logics_2023}. Le biais culturel pourrait au fond avoir des conséquences comparables au biais linguistique qui consiste à prendre trop peu de langues en considération (``language research shows a lack of openness to diverse languages and populations": \cite[23]{bochynska_reproducible_2023}). González Pérez ne mentionne pas les équivalents locaux (chinois ou yi) du terme `communauté', ni les spécificités des dynamiques sociolinguistiques locales. Politiques linguistiques \emph{(language policies)} et contexte politique ne font l'objet d'aucune mention, ce qui peut paraître paradoxal au vu de l'observation d'autres visiteurs étrangers selon laquelle la politique n'est nullement absente des préoccupations des gens dans l'Empire du milieu (lire par exemple Henri Michaux, \emph{Un barbare en Asie}, Paris: Gallimard, 1967, pages 178sq). Il va de soi que cette remarque ne constitue pas une critique visant l'auteur de l'article cité, mais simplement une observation concernant le risque toujours renaissant de généralisation indue au sujet de situations sur le terrain. Il existe un biais par lequel chacun a naturellement tendance à \emph{voir midi à sa porte} et à oublier que la diversité socio-géo-politique des situations est quasi-infinie.}. Le sentiment d'appartenance ethnique des Na constitue un domaine non seulement complexe (comme il va de soi dans les sociétés humaines) mais également sensible au plan politique, dans le contexte d'une Chine contemporaine dont les orientations résolument assimilationnistes visent à fondre les anciennes «nationalités minoritaires» (\pcmn{少数民族}) dans le creuset du «peuple chinois» (\pcmn{中华民族}). On ne cherchera pas ici à définir les frontières géographiques d'un dialecte na d'Alawua qui s'étendrait, au-delà de ce hameau, jusqu'à telle ou telle localité choisie selon des critères qui s'avéreraient \emph{in fine} manquer de consistance. On se contentera donc ici d'employer l'expression «le parler de Alawua» comme une simple étiquette pour décrire l'ensemble documentaire recueilli, sans prétendre en faire un dialecte défini avec précision au sein d'un paradigme de l'ensemble des dialectes na.

Le simple fait de doter ce parler d'un dictionnaire revient certes à lui faire une place de choix dans un paysage lexicographique contemporain encore très clairsemé. On peut espérer que l'orthographe dont est doté le dictionnaire contrebalance en partie cet effet, du fait qu'elle vise, elle, à un certain équilibre entre dialectes.



\section{Enquête de terrain et travail lexicographique: objectifs et chronologie}
\label{sec:chronologie}

%%Quelques notes au sujet des enquêtes de terrain sont présentées sur le site de la collection Pangloss, qui héberge les données recueillies\footnote{\url{https://pangloss.cnrs.fr/corpus/Yongning_Na?mode=normal&seeMore=true}}.

L'enquête de terrain présentait à l'origine une forte coloration phonologique, du fait que le système tonal de la langue attirait particulièrement l'attention. Pour autant, le travail s'inscrivait d'emblée dans la perspective de long terme d'un travail généraliste, dans la tradition de la linguistique dite \emph{de terrain}: recueillir des enregistrements audio, les transcrire sur place avec les locutrices et locuteurs, et élaborer, au fil du temps, un recueil de textes, un dictionnaire, et une grammaire.

Des séjours sur le terrain (au village) ont été effectués en 2006, 2007, 2008 et 2009, suivis en 2011-2012 d'un séjour d'un an dans la ville de Lijiang, où Mme Latami Daeshilamu avait déménagé. Des séjours plus courts ont été effectués en 2013, 2014, 2018 et 2024, toujours à la ville de Lijiang.

Une courte liste de mots a été dressée au moyen d'une élicitation à partir de mots chinois (expliqués tant bien que mal), puis progressivement complétée et corrigée au fur et à mesure de l'enregistrement et de la transcription de récits. Un inconvénient de ne pas éliciter directement de grandes quantités de vocabulaire est que l'enrichissement du dictionnaire par ajout de nouveaux mots est un processus lent. D'où le nombre limité d'entrées: de l'ordre de 3~000 (le nombre actuel est de \obtenircompteur{total}). Mais la collecte de parole spontanée présente cet avantage qu'un contexte est disponible pour aider à clarifier le sens des mots nouvellement rencontrés, offrant également une base pour une discussion plus approfondie de leur utilisation avec les consultants linguistiques \parencite[44]{mosel_dictionary_2004}.

\subsection{Le choix d'ouvrir les données par une publication en libre accès}

Dès les premières étapes du travail, le choix a été fait de placer les données en libre accès. Ce choix s'est avéré très favorable au développement du projet.

Une première occasion de partage public s'est offerte lorsque l'équipe du projet \emph{Sino-Tibetan Etymological Dictionary and Thesaurus} \parencite{stedt} a proposé d'héberger le lexique na telle qu'il se présentait à date de 2011. Un dépôt a donc été réalisé, en ajustant les données au format demandé.

La même année, sous l'impulsion de Guillaume Jacques et d'Aimée Lahaussois, un projet visant à la réalisation de dictionnaires et la collecte de corpus en langues himalayennes a été soumis à l'{Agence Nationale de la Recherche}. Ce projet, accepté en 2012, a été mis en œuvre à partir de 2013: il s'agit du projet HimalCo (ANR-12-CORP-0006). C'est dans ce cadre qu'a été entamée la transformation de la liste de mots en un véritable dictionnaire.

La collection Lexica, qui accueille le volume, est conçue et réalisée dans le cadre de la recherche publique. Elle s'inscrit dans une logique d'\emph{accès ouvert}. Pour reprendre les termes employés par Jean-Claude Guédon dans une discussion sur la liste internet du même nom («Accès ouvert»):

\begin{quotation}
    {\dots} une université se doit de défendre les processus de production des connaissances, particulièrement contre l'envahissement de ce territoire par des intérêts commerciaux. Règle générale, les doubles agendas~-- de connaissance et de commerce~--, quand ils sont impliqués dans la production des connaissances, n'augurent rien de bon pour celle-ci. La connaissance se fonde sur la discussion libre, incessante, sur la critique, le travail, et l'accès à l'archive des connaissances \emph{en jeu} dans la Grande Conversation. Le libre accès ne correspond à rien d'autre que de faciliter, améliorer, voire optimiser, les circonstances de la Grande Conversation. {\dots}

    Pour ceux et celles qui douteraient des effets pervers du monde commercial sur les publications savantes, la liste “Retraction Watch” est édifiante (\url{https://retractionwatch.com/}).

    (Jean-Claude Guédon, message du 13 septembre 2024, 15:41:09, sur «accesouvert - Liste de discussion de la communauté du libre accès francophone», animée et modérée par Jean-François Lutz et Pierre Mounier.)
\end{quotation}

\subsection{Chronologie et versions}

\label{sec:chronologie_etroite}
Les versions successives du travail sont numérotées, selon le modèle utilisé dans le développement de logiciels. Les versions produites à ce jour sont présentées ci-dessous par ordre chronologique. Un résumé sous forme de tableau est fourni dans le tableau \ref{table:versionsEN} pour les éditions en langue anglaise. Les éditions en chinois sont présentées dans le tableau \ref{table:versionsZH}, et les éditions en français dans le tableau \ref{table:versionsFR}.
\begin{longtblr}[
  caption = {Versions successives du dictionnaire dans sa mise en forme \emph{na-chinois français}},
  label = {table:versionsFR}
]{
  colspec = {X[0.8,l,m]X[0.8,l,m]X[l,m]X[1.4,l,m]},
  rowhead = 1
}
  \hline
  numérotation & date & lien vers HAL & bibliothèque logicielle \\
  \hline
  1.0 & septembre 2015 & \href{https://shs.hal.science/halshs-01204645v1/}{halshs-01204645v1} & \href{https://github.com/CNRS-LACITO/HimalCo/tree/master/dev/lib/pylmflib-1.1}{PyLMFlib} \\
  1.1 & novembre 2016 & \href{https://shs.hal.science/halshs-01204645v2/}{halshs-01204645v2} & \href{https://github.com/CNRS-LACITO/HimalCo/tree/master/dev/lib/pylmflib-1.1}{PyLMFlib} \\
  1.2.1 & avril 2018 & \href{https://shs.hal.science/halshs-01204645v3/}{halshs-01204645v3} & \href{https://github.com/CNRS-LACITO/Lexika}{Lexika (langage: Python)} \\
  2.0 & 2024 & \href{https://shs.hal.science/halshs-01204645v4/}{halshs-01204645v4} & \href{https://gitlab.com/BenjaminGalliot/JLexika}{JLexika (langage: Julia)} \\
  % \pnrubis{pò-ko-kabu} & \pnrubis{po-ko-kabun} & jeudi (3 j. [avant le] jour sacré) \\
  % \pnrubis{pò-tru kabu} & \pnrubis{po-ru-kabun} & vendredi (2 j. [avant le] jour sacré) \\
  % \pnrubis{po-xè kabu} & \pnrubis{po-xe kabun} & samedi (1 j. [avant le] jour sacré) \\
  \hline
\end{longtblr}

\begin{longtblr}[
  caption = {Versions successives du dictionnaire mis en forme pour un lectorat anglophone},
  label = {table:versionsEN}
]{
  colspec = {X[0.8,l,m]X[0.8,l,m]X[l,m]X[1.4,l,m]},
  rowhead = 1
}
  \hline
  numérotation & date & lien vers HAL & bibliothèque logicielle \\
  \hline
  1.0 & septembre 2015 & \href{https://shs.hal.science/halshs-01204638v1/}{halshs-01204638v1} & \href{https://github.com/CNRS-LACITO/HimalCo/tree/master/dev/lib/pylmflib-1.1}{PyLMFlib} \\
  1.1 & novembre 2016 & \href{https://shs.hal.science/halshs-01204638v2/}{halshs-01204638v2} & \href{https://github.com/CNRS-LACITO/HimalCo/tree/master/dev/lib/pylmflib-1.1}{PyLMFlib} \\
  1.2.1 & avril 2018 & \href{https://shs.hal.science/halshs-01204638v3/}{halshs-01204638v3} & \href{https://github.com/CNRS-LACITO/Lexika}{Lexika (langage: Python)} \\
  1.2.2 & juillet 2018 & \emph{pas de dépôt dans HAL} & \href{https://github.com/CNRS-LACITO/Lexika}{Lexika (langage: Python)} \\
  2.0 & 2024 & \href{https://shs.hal.science/halshs-01204638v4/}{halshs-01204638v4} & \href{https://gitlab.com/BenjaminGalliot/JLexika}{JLexika (langage: Julia)} \\
  % \pnrubis{pò-ko-kabu} & \pnrubis{po-ko-kabun} & jeudi (3 j. [avant le] jour sacré) \\
  % \pnrubis{pò-tru kabu} & \pnrubis{po-ru-kabun} & vendredi (2 j. [avant le] jour sacré) \\
  % \pnrubis{po-xè kabu} & \pnrubis{po-xe kabun} & samedi (1 j. [avant le] jour sacré) \\
  \hline
\end{longtblr}

\begin{longtblr}[
  caption = {Versions successives du dictionnaire mis en forme pour un lectorat sinophone},
  label = {table:versionsZH}
]{
  colspec = {X[0.8,l,m]X[0.8,l,m]X[l,m]X[1.4,l,m]},
  rowhead = 1
}
  \hline
  numérotation & date & lien vers HAL & bibliothèque logicielle \\
  \hline
  1.2 & mars 2018 & \href{https://shs.hal.science/halshs-01744420v1/}{halshs-01744420v1} & \href{https://github.com/CNRS-LACITO/Lexika}{Lexika (langage: Python)} \\
  2.0 & 2024 & \href{https://shs.hal.science/halshs-01744420v2/}{halshs-01744420v2} & \href{https://gitlab.com/BenjaminGalliot/JLexika}{JLexika (langage: Julia)}\\
  \hline
\end{longtblr}

% \begin{longtblr}[
%   caption = {Versions successives du dictionnaire mis en forme pour un lectorat anglophone},
%   label = {table:versionsEN}
% ]{
%   colspec = {X[1,l,m]X[l,m]X[2.5,l,m]},
%   rowhead = 1
% }
%   \hline
%   numérotation & date & lien vers HAL \\
%   \hline
%   1.0 & septembre 2015 & \url{https://shs.hal.science/halshs-01204638v1/} \\
%   1.1 & novembre 2016 & \url{https://shs.hal.science/halshs-01204638v2/} \\
%   1.2.1 & avril 2018 & \url{https://shs.hal.science/halshs-01204638v3/} \\
%  1.2.2 & juillet 2018 & \emph{pas de dépôt dans HAL} \\
%   2.0 & 2024 & \url{https://shs.hal.science/halshs-01204638v4/} \\
%   \hline
% \end{longtblr}

% \begin{longtblr}[
%   caption = {Versions successives du dictionnaire mis en forme pour un lectorat sinophone},
%   label = {table:versionsCN}
% ]{
%   colspec = {X[1,l,m]X[l,m]X[2.5,l,m]},
%   rowhead = 1
% }
%   \hline
%   numérotation & date & lien vers HAL \\
%   \hline
%   1.2 & mars 2018 & \url{https://shs.hal.science/halshs-01744420v1/} \\
%    2.0 & 2024 & \url{https://shs.hal.science/halshs-01744420v2/} \\
%   \hline
% \end{longtblr}

\subsubsection{Versions 1.0 et 1.1: encodage conforme à la norme LMF (Lexical Markup Framework)}

Lors d'une première étape du projet, l'équipe a adopté la norme LMF \emph{(Lexical Markup Framework)}, un format pivot conçu pour des dictionnaires qui se prêtent à des traitements informatiques \parencite{francopoulo2013,romary2013}. Céline Buret, ingénieure en informatique qui a travaillé avec l'équipe du projet pendant deux ans (nov. 2014-oct. 2015), a réalisé des scripts de conversion des données lexicographiques, depuis le format Toolbox (Multi-Dictionary Formatter, MDF) vers un format XML conforme à la norme LMF. Une bibliothèque Python 2 a été développée: PyLMFlib, pour \emph{Python LMF library}. En 2015, la version 1.0 des versions en ligne et PDF du dictionnaire a été publiée, ainsi que la base de données source au format MDF (Toolbox). La version 1.1, publiée en 2016, utilisait la même bibliothèque de scripts informatiques.

\subsubsection{Le logiciel Lexika et la collection de dictionnaires Lexica}

Une limite du formalisme de balisage lexical \emph{Lexical Markup Framework} (LMF) dans sa formulation de 2013 est progressivement apparue: il impose une contrainte sur ce qu'une entrée peut contenir. Les sous-entrées qui appartiennent à des catégories grammaticales différentes doivent faire l'objet d'entrées distinctes. Par exemple, \phonologie{lɑ˧-kʰv̩˧˥} peut signifier à la fois ‘année du Tigre’ et ‘né l'année du Tigre’ (en chinois: \pcmn{虎年} et \pcmn{属虎}). Or, du point de vue du linguiste, il peut être souhaitable de créer deux sous-entrées au sein d'une même entrée. Mais les catégories de partie de discours sont différentes: ‘année du tigre’ est un syntagme nominal, et ‘né l'année du tigre’ est un prédicat, catégorisé comme un adjectif. Dans le cadre du \emph{Lexical Markup Framework}, cette situation nécessite la création de deux entrées différentes, ce qui est contraire à la pratique lexicographique courante.

Fin 2016, Benjamin Galliot, travaillant au CNRS-LACITO dans le cadre d'un contrat à durée déterminée (six mois), a écrit une nouvelle bibliothèque, Lexika\footnote{Le nom du logiciel, \emph{Lexika}, diffère (dans sa forme écrite) de celui choisi pour la série de dictionnaires de la collection Pangloss: \emph{Lexica} \parencite{lexica2017}. Il s'agit d'éviter toute confusion entre le logiciel d'une part et la série de dictionnaires d'autre part.}, en utilisant la version 3 du langage Python, qui, entre autres avantages, gère nativement le codage Unicode. Consigne avait été donnée à l'époque de s'en tenir à la norme \emph{Lexical Markup Framework} (LMF), bien que Benjamin Galliot en ait mesuré très rapidement les limites, et ait exprimé le souhait de repartir à zéro sur des bases moins contraintes. Un argument pour ne pas changer ce choix était la brièveté de la durée du contrat de Benjamin Galliot, qui paraissait incompatible avec la refonte ambitieuse que celui-ci souhaitait concevoir et mettre en œuvre.

Mi-2017, Benjamin Galliot, libre de tout employeur, a développé à titre personnel une refonte totale de Lexika \parencite{galliot:2017:lexika}, longtemps appelée JLexika – car développée en langage Julia plutôt que Python –, en guise de défi personnel et d’apprentissage d’un langage qui l’intéressait et qui épousait davantage sa manière de programmer. Certaines des problématiques soulevées au fil de ses travaux, et des solutions développées pour y répondre, sont exposées dans \cite{galliot:2023:lexikaproblématiques}.

Fin 2017 (durant un très court contrat, de moins de trois mois), à la lumière d'échanges avec Laurent Romary et Mathieu Mangeot-Nagata, il a été conclu que l'adhésion à la norme LMF n'apportait décidément pas d'avantages clairs, tandis qu'elle nécessitait la mise en œuvre de procédures complexes afin de contourner certaines limites imposées par cette norme. Comme à ce moment-là, les travaux personnels de Benjamin Galliot sur JLexika étaient partis sur la base d’un format libre de toute norme, par mimétisme, le modèle XML de (P)Lexika s’est grandement éloigné de la norme LMF début 2018 pour répondre aux attentes et aux souhaits
%je proposerais cette petite reformulation, plutôt que "suivre les besoins voire désirs": affaire de style personnel. Tous choix stylistiques étant entièrement de ton ressort pour une future postface que je t'encouragerais à écrire à ta manière, si les circonstances y sont favorables. Il va de soi que le texte de cette intro pourrait alors être revu en fonction, si tu le juges opportun, par exemple en déplaçant vers la postface certains éléments qui touchent plus particulièrement à ton travail, tels que la discussion autour de LMF.
des linguistes. Une préoccupation constante au fil de ce travail était de conserver une grande cohérence interne, afin d'éviter tout effet « lit de Procuste » par lequel les pratiques lexicographiques devraient se plier, sans véritable nécessité interne, à des contraintes imposées \emph{a priori}.

Des années après, lorsque Benjamin Galliot a intégré à nouveau le CNRS comme fonctionnaire, il a décidé qu’il était plus pertinent de continuer le développement de JLexika, version qui était bien plus puissante et robuste que PLexika. La philosophie de Lexika, poussée à l’extrême dans sa dernière mouture, peut se résumer par la formule « Comment modéliser la création de dictionnaires eu égard à la diversité des langues et des linguistes ? » Le format \emph{maison} est personnalisable à souhait pour chaque dictionnaire (avec une base commune aux dictionnaires de la collection Lexica\footnote{\url{https://hal.science/LEXICA}}). Ainsi, bien qu’il soit compatible avec la norme de la \emph{Text Encoding Initiative} (TEI) – par le truchement d’une feuille de style XSL –, sa raison d’être réside dans une personnalisation poussée pour une adaptation maximale à la langue et au linguiste. Entre autres particularités, il est possible d’avoir la hiérarchie souhaitée (sous-entrées et/ou polysémie, subdivisions diverses, etc.), du texte enrichi à tous les niveaux (styles divers, liens, mélange de langues et scripts, etc.) ainsi que toutes les informations jugées nécessaires (notes/commentaires, étiquettes, informations encyclopédiques, explications au sujet de l'étymologie), pouvant se rattacher à divers niveaux de la structure.

%Le modèle XML a été refondu à la lumière d'échanges avec Laurent Romary et Mathieu Mangeot-Nagata, qui ont permis de conclure que l'adhésion à la norme LMF n'apportait pas d'avantages clairs, tandis qu'elle nécessitait la mise en œuvre de procédures complexes afin de contourner certaines limites imposées par cette norme. Entre autres avantages, le nouveau format facilite les références croisées entre les entrées, par exemple en indiquant les synonymes et antonymes.

Dans la chaîne de traitement Lexika, les versions PDF du dictionnaire sont générées à partir du fichier XML, lui-même généré à partir du fichier source, qui demeure au format MDF.

Des versions intermédiaires sont générées au fur et à mesure de l'avancement du travail lexicographique et mises à disposition via le dépôt GitHub qui héberge la base de données\footnote{\url{https://github.com/alexis-michaud/na/tree/master/DICTIONARY}}.

\section{Guide d'utilisation}
\label{sec:guide}

La série Lexica vise à joindre la lisibilité pour les utilisatrices et utilisateurs humain·es à un encodage lisible par ordinateur (adapté au Traitement Automatique des Langues Naturelles). Les dictionnaires sont donc proposés dans plusieurs formats:
\begin{itemize}
    \item sous forme de documents PDF tels que le présent document
    \item sous forme de dictionnaires en ligne au format HTML
    \item sous la forme d'une bases de données au format Toolbox/MDF (fichier principal)
    \item sous forme de fichiers XML produits par le logiciel Lexika.
\end{itemize}

%La base de données est disponible sur le dépôt GitHub\footnote{\url{https://github.com/alexis-michaud/na/tree/master/DICTIONARY}} en deux formats:

La base de données est disponible sur le dépôt GitHub sous deux formats:

\begin{itemize}
    \item un fichier XML utilisant le modèle conçu par Benjamin Galliot
    \item le fichier maître au format MDF, qui peut être ouvert avec le logiciel Toolbox ou avec un éditeur de texte.
\end{itemize}

La base de données Toolbox est quadrilingue: les mots et exemples sont traduits en français, chinois et anglais. Pour le dictionnaire au format PDF, trois mises en forme sont proposées: le présent document, na-chinois-français (\hyperlink{https://shs.hal.science/halshs-01204645/}{halshs-01204645}), un second destiné au lectorat anglophone (\hyperlink{https://shs.hal.science/halshs-01204638}{halshs-01204638}), et un troisième destiné au public sinophone (\hyperlink{https://shs.hal.science/halshs-01744420}{halshs-01744420}).

Après une présentation de la structure de la base de données, des explications seront fournies au sujet des choix typographiques.

\subsection{Structure de la base de données}

L'extrait du code XML présenté ci-dessous répond par lui-même à l'essentiel des questions concernant la structure de la base de données.
\begin{lstlisting}[language=XML, caption=Extrait du code XML Lexika illustrant la structure des entrées, label=code:LexikaXML]
<EntréeLexicale identifiant="ⓔhĩ˧-ʈʂɤ#˥">
  <Lemme>
    <Forme>hĩ˧-ʈʂɤ#˥</Forme>
    <Ton>#H</Ton>
    <Orthographe>hinzhe</Orthographe>
    <FormeDeSurface>hĩ˧ʈʂɤ˧</FormeDeSurface>
  </Lemme>
  <PartieDuDiscours langue="eng">n</PartieDuDiscours>
  <ListeDeSens>
    <Sens>
      <DomainesSémantiques>
        <DomaineSémantique langue="fra">société</DomaineSémantique>
        <DomaineSémantique langue="eng">society</DomaineSémantique>
      </DomainesSémantiques>
      <Définitions>
        <Définition langue="eng">Family member belonging to the same lineage (on the mother's side).</Définition>
        <Définition langue="cmn">亲戚:有共同祖先(祖母)的家人</Définition>
        <Définition langue="fra">Membre de la famille de même lignage (du côté maternel).</Définition>
      </Définitions>
      <Gloses>
        <Glose langue="eng">family_members_of_same_lineage</Glose>
        <Glose langue="cmn">亲戚</Glose>
        <Glose langue="fra">membre_de_la_famille_de_même_lignage</Glose>
      </Gloses>
      <Exemples>
        <Exemple>
          <Original langue="nru">hĩ˧-tɕʰɯ˧ - hĩ˧-ʈʂɤ#˥</Original>
          <Traduction langue="eng">same meaning: the family members belonging to the same lineage</Traduction>
          <Traduction langue="cmn">同上:亲戚,有共同祖先(祖母)的家人</Traduction>
          <Traduction langue="fra">même sens: les gens de même lignage</Traduction>
        </Exemple>
      </Exemples>
      <RelationsSémantiques>
        <RelationSémantique>
          <Cible identifiant="ⓔhĩ˧-tɕʰɯ#˥">hĩ˧-tɕʰɯ#˥</Cible>
          <Type langue="fra">synonyme</Type>
        </RelationSémantique>
      </RelationsSémantiques>
    </Sens>
  </ListeDeSens>
</EntréeLexicale>
\end{lstlisting}

Le dictionnaire est composé d'\emph{entrées lexicales}. Chacune est pourvue d'un identifiant informatique. Celui-ci n'est pas affiché dans la version PDF du dictionnaire, sa fonction étant de permettre des manipulations par ordinateur, et non d'être reconnu par des yeux humains. Cet identifiant est composé du caractère spécial \textecode{ⓔ} suivi de la représentation phonologique du mot (ici, \phonologie{hĩ˧-ʈʂɤ\#˥}), avec un système de notation des tons qui recourt à des symboles spéciaux (\$ et \#) pour distinguer les diverses sortes de tons hauts, en fonction de leur mode d’association avec les syllabes. Toutes explications au sujet de ce système sont fournies dans la monographie \emph{Tone in Yongning Na}, disponible en ligne en libre accès \parencite[80-90]{michaud2017}.

Le \emph{lemme} de l'entrée lexicale comprend quatre informations: sa forme phonologique abstraite, son ton, une représentation orthographique (fournie par Roselle Dobbs), et enfin sa forme de surface (qui correspond à sa réalisation lorsque le mot est dit isolément).

La partie du discours (classe morphosyntaxique) est ensuite fournie.

Commence ensuite la description du ou des sens du mot. Une indication du domaine sémantique est fournie, en choisissant au sein d'une liste fermée de termes (en français et en anglais); cette information ne figure pas, à l'heure actuelle, dans le dictionnaire PDF. Des définitions sont fournies en anglais, chinois et français. Dans quelques cas, une définition en langue vernaculaire est fournie. Le procédé est d'une grande utilité \parencite{dingemanse_folk_2015}, et la co-autrice du dictionnaire sait le pratiquer; s'il n'a pas été généralisé jusqu'ici, c'est simplement faute de temps.

Des gloses sont également fournies, en vue d'un futur glosage systématique de textes. Ces gloses ne figurent pas dans le PDF, du fait qu'elles sont redondantes avec la définition (et moins informatives).

Les exemples sont également considérés comme faisant partie du bloc «Sens», dans la mesure où ils illustrent l'un des sens du mot. Chaque exemple comprend une transcription et une traduction dans les trois langues-cibles. De nombreux exemples sont accompagnés de notes, lesquelles comprennent un attribut qui indique le \emph{domaine} linguistique concerné: sémantique, syntaxe, morphologie, phonologie, tonologie\footnote{(La tonologie est codée différemment de la phonologie du fait de la place importante qu'elle tient dans le travail de documentation et recherche réalisé.}, dialectologie. Ces informations complémentaires apparaissent dans le fichier PDF (sauf exception signalée par une étiquette \textecode{print=\textquotedbl n\textquotedbl}).

% sem:Sémantique Semantics 语义
% synt:Syntaxe Syntax 句法
% morpho:Morphologie Morphology 形态
% phono:Phonologie Phonology 音系学
% tone:Tonologie Tonology 声调
% dialect:Dialectologie Dialectology 方言学

Une étiquette «historique» (\textecode{type=\textquotedbl hist\textquotedbl}) associée à une note signale des explications qui concernent l'historique du travail au sujet du mot concerné, des premières enquêtes sur le terrain jusqu'à la version en cours. Les notes d'historique relatent les tâtonnements dans les premières notations, indiquent la date d'adoption de tel ou tel changement dans la notation, et détaillent les arguments ayant joué en faveur de tel ou tel choix. Par exemple, l'entrée \phonologie{ŋwɤ˧pʰæ˧˥} ‘tuile’ comporte une note indiquant que le mot avait été initialement transcrit avec un schéma tonal M.H, et avec un timbre vocalique \phonétique{æ} dans ses deux syllabes: \phonétique{ŋwæ˧pʰæ˥}. La note explique que la perception de \phonétique{æ} dans la première syllabe est due à une tendance phonétique à l'harmonie vocalique régressive. Le résultat de vérifications est également consignée dans ces notes, dont sont pourvues environ la moitié des entrées. Cet historique parfois tortueux ne paraissait pas pertinent pour la majorité des lectrices et lecteurs, de sorte qu'il n'apparaît généralement pas dans la version PDF.

Les relations sémantiques telles que la synonymie font l'objet d'un encodage, de même que la composition morphologique. Cela permet de relier un mot disyllabique ou polysyllabique et les morphèmes dont il est constitué: par exemple pour signaler que \phonologie{æ˧ʂæ˧-pi˧mv̩˧˥} ‘conte’ est formé des deux mots \phonologie{æ˧ʂæ\#˥} ‘autrefois’ et \phonologie{pi˧mv̩˥\$} ‘dicton, adage’.

%Les commentaires fournis dans les 3 langues ont vocation à être de simples traductions: à terme, toutes les infos (généralement rédigées à l'origine en français) seront traduites dans les 3 langues, de sorte qu'en principe il ne sera pas nécessaire d'avoir les commentaires dans les 3 langues.

%Test de mot chinois: \pcmn{您好} !
\subsection{Structure des entrées dans la version mise en page (PDF)}
\label{sec:structure_des_entrées}

Dans la version mise en page, la représentation phonologique du mot (ici, \phonologie{ɑ˩pʰv̩\#˥}) figure en vedette. Comme expliqué précédemment, le système de notation des tons qui recourt à des symboles spéciaux (\$ et \#) pour distinguer les diverses sortes de tons hauts, en fonction de leur mode d’association avec les syllabes \parencite[80-90]{michaud2017}.

Ensuite vient, entre barres obliques, la prononciation du mot à l’isolée (ici, \phonétique{ɑ˩pʰv̩˥}): ce qui est couramment appelé la «forme de citation» du mot. Celle-ci permet au lecteur familier de l’Alphabet Phonétique International de connaître la prononciation du mot. Dans le cas des classificateurs, qui d’ordinaire n’apparaissent pas seuls, on a fourni leur réalisation lorsqu’ils sont associés au numéral «un».

Le troisième champ est une représentation orthographique (ici: \emph{opu}), dans un système conçu par Roselle Dobbs et Xióng Yàn \parencite[]{dobbs_ortho_2018}. Cette orthographe, intégrée au dictionnaire depuis sa version 1.2, s’inspire de la romanisation \emph{Pinyin} du chinois mandarin. Il ne s’agit pas d’une translittération de la notation en alphabet phonétique: le dictionnaire est mono-dialectal, tandis que l’orthographe a été discutée avec des locuteurs de divers dialectes afin de rechercher un compromis qui soit acceptable dans diverses localités. Ainsi, parmi les tons, qui varient nettement d’un dialecte à l’autre, seul le ton bas est indiqué dans l’orthographe.

Les similarités et différences idiolectales de prononciation sont ensuite indiquée, en associant le code locuteur à l'étiquette ``idem" (dans le cas où la prononciation du locuteur concerné est identique à celle de la co-autrice du dictionnaire, locutrice de référence) ou à une transcription en Alphabet Phonétique International de leur prononciation. Le cas échéant, des notes sont fournies au sujet des variantes. Chaque note est dotée d'une étiquette indiquant le domaine linguistique concerné, selon la même nomenclature que les notes qui se rattachent à l'entrée dans son ensemble: sémantique, syntaxe, morphologie, phonologie, tonologie, dialectologie, étymologie, comparatisme (phonologie historique), diachronie (réflexions historiques qui ne constituent, ni une étymologie du mot en question, ni une information concernant des cognats ou comparanda dans d’autres langue), typologie, usage (pragmatique).

Suit la classe morphosyntaxique (nom, verbe, etc.), indiquée au moyen d'une des étiquettes dont la liste figure dans le Tableau~\ref{table:PartsOfSpeech}.

\begin{longtblr}[
  caption = {Parties du discours (classes morphosyntaxiques)},
  label = {table:PartsOfSpeech}
]{
  colspec = {X[l,m]X[l,m]X[l,m]},
  rowhead = 1
}
  \hline
  {abréviation} & {sens} & {nombre d'entrées dans le dictionnaire}  \\
  \hline
        \textsc{adj} & adjectif & \obtenircompteur{adj} \\
        \textsc{adv} & adverbe & \obtenircompteur{adv} \\
        \textsc{clf} & classificateur & \obtenircompteur{clf} \\
        \textsc{clitic} & clitique & \obtenircompteur{clitic} \\
        \textsc{cnj} & conjonction & \obtenircompteur{cnj} \\
        \textsc{disc.ptcl} & particule discursive & \obtenircompteur{disc.ptcl} \\
        \textsc{ideophone} & idéophone & \obtenircompteur{ideophone} \\
        \textsc{intj} & interjection & \obtenircompteur{intj} \\
        \textsc{n} & nom & \obtenircompteur{n} \\
        \textsc{num} & numéral & \obtenircompteur{num} \\
        \textsc{postp} & postposition & \obtenircompteur{postp} \\
        \textsc{pref} & préfixe & \obtenircompteur{pref} \\
        \textsc{prep} & préposition & \obtenircompteur{prep} \\
        \textsc{pro} & pronom & \obtenircompteur{pro} \\
        \textsc{suff} & suffixe & \obtenircompteur{suff} \\
        \textsc{v} & verbe & \obtenircompteur{v} \\
  \hline
\end{longtblr}

La classe tonale du mot est indiquée à droite de l'indication concernant la partie du discours. La classe tonale indiquée est la catégorie phonologique sous-jacente à laquelle le mot appartient. Cette information est déjà présente dans la transcription phonologique, mais pas de façon transparente, de sorte que le fait de la répéter séparément paraissait de nature à faciliter l'accès à cette information.

Les mots archaïques sont signalés comme tels par la mention «Archaïque».

Suivent la traduction chinoise et la traduction française. La base de données lexicographique comporte également une traduction en anglais, mais la présence du français et de l’anglais côte à côte, même distingués par la typographie, paraissait gêner la lecture pour des lecteurs francophones ou anglophones, ce qui a amené à exclure l’anglais de la présente déclinaison du dictionnaire. En revanche, les traductions chinoises sont souvent un complément utile à la traduction française, car il existe souvent des équivalents plus proches. Par exemple, \phonologie{gɤ˧˥} se traduit par le simple \pcmn{扛} en chinois, alors que la traduction française est moins simple: «porter sur l'épaule».

Les exemples (expressions et phrases illustrant l’usage du mot) commencent par une marque \textecode{¶}; certains comportent une mention «(Proverbe)», d'autres, élicités pour étudier les tons de combinaisons spécifiques entre mots, ont une mention «(Élicitation phonologique)». La source des exemples est indiquée, sauf lorsqu'il s'agit d'informations fournies par la locutrice de référence.

Les notes apparaissent ensuite, précédée chacune par une indication du domaine concerné: sémantique, syntaxe, morphologie, phonologie, tonologie, dialectologie, étymologie,  comparatisme (phonologie historique), diachronie (réflexions de linguistique diachronique qui ne constituent, ni une étymologie du mot en question, ni une information concernant des cognats ou comparanda dans d’autres langues), typologie, usage (pragmatique).

Enfin viennent les renvois (synonymes, antonymes…) et, pour un nom, les classificateurs qui lui sont le plus couramment associés (ici, \phonologie{mi˩}).

Certaines racines monosyllabiques extraites de disyllabes sont indiquées par le symbole †. Aucune forme de surface n'est fournie, car ces formes monosyllabiques ne sont pas actuellement utilisées en tant que telles dans la langue.

Les emprunts au chinois et au tibétain sont indiqués dans les cas où leur identification paraît tout à fait claire. Il n'y a pas eu d'élicitation systématique visant à recueillir des emprunts à l'une ou l'autre langue, mais les mots empruntés apparaissant dans les textes sont ajoutés à mesure au dictionnaire. Les informations fournies comprennent: la langue d'origine, la forme dans la langue d'origine, et parfois une note explicative.

\subsection{Liens vers les textes en ligne dans la collection Pangloss}

Comme évoqué au §\ref{sec:structure_des_entrées}, certains des exemples proviennent du corpus de textes en langue na (enregistrements audio transcrits) de la collection Pangloss, disponible en ligne en libre accès \parencite[voir][]{michailovskyetal2014}. Un lien en un clic du PDF vers l'exemple dans son contexte d'origine est proposé, au moyen de l'identifiant d'objet numérique (DOI) du document. Pour d'autres exemples, qui semblent particulièrement éclairants pour l'étude du mot en question mais qu'il ne paraît pas très pertinent de reproduire intégralement dans le dictionnaire, une référence et un lien hypertexte sont fournis.

Au-delà de ces liens intégrés au dictionnaire de façon ponctuelle, il serait souhaitable de créer à l'avenir des liens \emph{systématiques} et dynamiques entre les ressources, afin que les dictionnaires, la documentation linguistique (en premier lieu les textes, qui constituent le cœur des ressources) et les travaux publiés au sujet de la langue (articles, grammaires…) puissent être interconnectés de façon plus étroite \parencite{maxwell2012}. Les occurrences textuelles constituent en fin de compte la meilleure ressource pour documenter l'usage d'un mot. Les exemples fournis dans le dictionnaire à l'heure actuelle sont peu nombreux en comparaison des occurrences dans les textes, et leur contexte d'utilisation peut manquer de clarté, en dépit des efforts déployés pour expliciter le contexte des exemples notés au cours du travail sur le terrain.

\begin{quotation}
    C'est que quand nous considérons un énoncé, nous tendons à l'intoner intérieurement, d'où des différences dont nous n'avons pas conscience, à le situer dans des contextes implicites qui entremêlent la plausibilité sémantique (ou pragmatique) et l'acceptabilité grammaticale. Enfin, tout énoncé appartient à une famille paraphrastique, où il nous arrive de glisser d'un énoncé à un énoncé équivalent, mais entraînant une modulation différente. \parencite[17]{culioli1990}
\end{quotation}

Il n'est à l'heure actuelle pas possible de fixer un calendrier prévisionnel concernant la réalisation de concordances complètes des occurrences dans les textes. Cette avancée nécessitera de gloser les textes au niveau du mot et de déployer des identifiants (tels que ceux utilisés dans la base de données lexicale) afin de relier sans erreur les mots des textes aux entrées du dictionnaire.




%repose sur des enquêtes de première main, réalisées sur le terrain au village en 2006, 2007, 2008 et 2009, puis à la ville de Lijiang, où s'était établie Mme Latami Daeshilamu (locutrice de référence), en 2011 et 2012, puis plus brièvement en 2013, 2014 et 2018. La langue est présentée dans une monographie au sujet du système tonal de ce parler \parencite[2-8]{michaud2017}. Quelques notes au sujet des enquêtes de terrain sont présentées sur le site de la collection Pangloss, qui héberge les données recueillies\footnote{\url{https://pangloss.cnrs.fr/corpus/Yongning_Na?mode=normal&seeMore=true}}.



\section{Perspectives de recherche}
\label{sec:recherche}

\subsection{Lexicologie et analyse linguistique}

Les enjeux de la lexicographie pour la recherche en linguistique ont parfois été sous-estimés, de sorte que la lexicographie a pu faire figure de parent pauvre au sein des sciences du langage. On se souvient du propos de Samuel Johnson, qui déplore, dans la préface de son \emph{Dictionnaire de la langue anglaise} de 1755, le peu de reconnaissance accordé aux lexicographes.

\begin{quotation}
    It is the fate of those who toil at the lower employments of life, to be rather driven by the fear of evil, than attracted by the prospect of good; to be exposed to censure, without hope of praise; to be disgraced by miscarriage, or punished for neglect, where success would have been without applause, and diligence without reward.

    Among these unhappy mortals is the writer of dictionaries; whom mankind have considered, not as the pupil, but the slave of science (…). Every other authour \emph{(sic.)} may aspire to praise; the lexicographer can only hope to escape reproach, and even this negative recompence \emph{(sic.)} has been yet granted to very few.
\end{quotation}

On pourrait voir là un exemple d'auto-dépréciation \emph{(self-deprecation)} tout britannique, mis au service d'une stratégie rhétorique visant à s'attirer la sympathie des lectrices et lecteurs (la \emph{captatio benevolentiae} de la rhétorique classique), si le propos n'était confirmé par les propos de linguistes, y compris des contemporains. Deux siècles et demi plus tard, un professeur de linguistique anglais formulait de façon lapidaire une observation qui allait bien dans le même sens: “the lexicographer is the equivalent in linguistics to the guy stacking the shelves at Sainsbury’s” («le lexicographe, c'est l'équivalent, en linguistique, du gars qui remplit les rayons au supermarché»; communication personnelle anonymisée, 1996). %À l'opposé de l'analyse de la morphosyntaxe d'une langue, qui ambitionne de mettre en évidence la cohérence du système grammatical dans son ensemble, la description du lexique serait, par nature, parcellaire.
Dans cette vision simpliste, la description de la morphosyntaxe d’une langue vise à mettre en lumière le système grammatical dans son ensemble, à déceler les liens qu’entretiennent ses diverses composantes, tandis que la description du lexique serait nécessairement parcellaire. L’ordre alphabétique dans lequel se présente le dictionnaire vaudrait aveu de l’absence d’organisation du lexique.

Se laisser gagner par ce point de vue, c'est se priver de découvrir que la lexicographie a vocation à déboucher sur une \emph{lexicologie}. Rédiger un dictionnaire revient à explorer la \emph{structure du lexique} d'une langue \parencite{francois2008semantic}, en lien avec l'étude des structures sociales et culturelles. La rédaction d'un dictionnaire nécessite de creuser le sens, de tenter de cerner les connotations, la polysémie, les relations entre les mots. Certaines entrées de dictionnaire se prêteraient à un développement sous forme d'articles de recherche.

Ainsi, le vocabulaire de la parenté fournit des informations précieuses sur les structures familiales et leur histoire. Dans le cas de la langue Na, cette source d’information a été bien identifiée \parencite{fu1983} mais reste à exploiter de façon systématique, en rapport avec l'étude de la dynamique des liens sociaux \parencite{milan2021entraide}. Le domaine de la parenté fait figure de grand classique du genre, mais des travaux ont montré la voie pour l’exploration d’autres domaines comme celui des émotions \parencite{tersis_langage_2017} ou de la spiritualité \parencite{francois_shadows_2013}. Sur la base de descriptions lexicographiques approfondies peuvent se déployer diverses approches de recherche, qui combinent la typologie avec une dimension dynamique (diachronique).

L'étude des effets du contact sur le lexique constitue un autre axe de recherche prometteur.

\begin{quote}
« La tendance, chez les individus bilingues, à aligner les structures sémantiques des langues qu’ils parlent, a permis la diffusion de certaines catégorisations lexicales à l’échelle de vastes aires linguistiques et culturelles: c’est ainsi que certains découpages sémantiques, certaines polysémies ou phraséologies, deviennent les symptômes d’une aire donnée. Parfois, il est possible d’expliquer ces phénomènes aréaux par des liens entre pratiques langagières et pratiques sociales répandues dans la région: certains modes d’organisation familiale, par exemple, pourront être corrélés à des structures lexicales spécifiques dans le domaine de la parenté, ou dans le vocabulaire du mariage et des relations interpersonnelles. » (Alexandre François et Lameen Souag, séminaire « Structures du lexique: typologie et dynamiques », LACITO, janvier 2018)
\end{quote}

En particulier, l'étude comparée du lexique du na et du pumi \parencite{daudey2014} est particulièrement prometteuse pour une compréhension approfondie de la langue et de la culture de ces deux peuples qui coexistent à Yongning. C'est l'une des nombreuses questions de recherche que le présent dictionnaire devrait permettre d'étudier. Grâce au travail de Gerong Pincuo, des termes de comparaison (comparanda) en langue pumi sont proposés pour un certain nombre de mots du dictionnaire (de l'ordre d'une centaine au stade de la présente version): Gerong Pincuo \pcmn{给汝品初} a parcouru l'ensemble de la liste de mots (dans son état de 2012) et fourni le terme pumi lorsque celui-ci lui paraissait comparable. Il peut s'agir d'un emprunt fait par le na et le pumi à une troisième langue, telle que le tibétain (en particulier pour le vocabulaire religieux) ou le chinois, ou de mots qui auraient été empruntés au na par le pumi (ou l'inverse), ou encore de vocabulaire hérité.

Les quelques paragraphes ci-dessous élargissent la réflexion au-delà du cadre de la linguistique proprement dite, pour relier la lexicographie à des questionnements issus d'autres disciplines.

\subsection{Le lexique et ses marges}
\label{sec:marges}
Jacques Lacan désigne par \emph{linguisterie} l'ordinaire du travail des linguistes, dégageant au moyen de ce néologisme l'espace d'une linguistique élargie, qui ne serait pas exclusivement celle des linguistes: une linguistique mieux ouverte à un dialogue épistémologique avec d'autres sciences. Si l'on se prend au jeu de cette proposition, on est amené à dessiner pour la lexicographie, autour de la \emph{lexicographisterie} ordinaire (pardon pour la surenchère dans la création de néologismes), un domaine élargi, et des pratiques lexicographiques exigeantes qui veillent à ne pas se \emph{payer de mots}. Lever les yeux de son dictionnaire est un geste qu'il faut sans cesse renouveler, afin de se rappeler l’enjeu épistémologique et humain de l’entreprise qui consiste à décrire une langue en voie de disparition.

Au fil du processus de création d'un dictionnaire, on est amené à prendre des décisions de divers ordres: fixer l'inventaire des classes morphosyntaxiques, décider de séparer ou regrouper deux nuances de sens après en avoir soupesé le degré de différenciation, choisir tel exemple plutôt que tel autre pour illustrer un certain emploi. Ce travail de \emph{lexicographisterie} comporte une latitude de bricolage (lexico-rafistolages, lexicographistolages?), qu'il importe de ne pas se laisser aller à prendre pour argent comptant. L'enjeu est que le projet se maintienne \emph{en tension} vers son objet (la langue), sans verser dans l'ornière d'un fonctionnement dont la belle cohérence se paierait au prix de l'adéquation à son véritable objet. Dans cette perspective, l'étude de composantes du lexique souvent considérées comme marginales peut jouer un rôle moteur.

Au cours des premières étapes du travail de description d'une langue rare, il s'agit de dégager les structures centrales. De même qu'en syntaxe il est de bonne méthode de commencer par des constructions simples, pour ensuite étudier leur réélaboration au sein de structures plus complexes, de même le travail lexicographique s'est-il d'abord concentré sur la tâche qui consistait à réunir des noms et verbes en nombre suffisant pour étudier la phonologie, et pour fournir matière à un débroussaillage de la phonologie historique \parencite{jacquesetal2011}. Une attention particulière a ensuite été prêtée aux classificateurs \parencite{michaud2011c,michaud2013d}, de sorte que cette classe morphosyntaxique n'est pas trop mal représentée dans le dictionnaire (136 entrées), mais les autres parties du discours demeuraient des parents pauvres. L'étude du système tonal, ainsi que les collaborations avec des informaticien·nes en vue de la transcription automatique d'enregistrements audio en langue na \parencite{michaudetal2018integrating,guillaume_plugging_2022} en vue d'avancer dans la constitution d'un corpus de textes conséquent, ont occupé mon temps de recherche disponible jusqu'en 2024.

C'est donc seulement de façon assez tardive que l'attention s'est portée sur les accotements plus ou moins stabilisés du lexique (ce que le vocabulaire technique routier anglais désigne comme des \emph{soft shoulders}). En particulier, emprunts, onomatopées et idéophones n'ont été intégrés que tardivement. À partir de la version 2, le dictionnaire est instruit par une vision du lexique qui prête attention à ces composantes précédemment négligées.

Les praticien·nes de l'ethnolinguistique s'intéressent au lexique et à sa motivation dans les langues les plus diverses \parencite{deColombel2002lexique}. Les phénomènes expressifs paraissent particulièrement abondants dans les langues d'Asie:

\begin{quotation}
    The languages of Mainland Southeast Asia are resplendent with elaborate grammatical resources for fashioning elaborative expressions that convey emotions, senses, conditions, and perceptions that enrich discourse -- both everyday and ritualized -- and are grammatical works of art. Over time, a sizeable terminological lexicon has been created to categorize or classify these resources, including echo words, phonaesthetic words, chameleon affixes, chiming derivatives, onomatopoeic forms, ideophones, and most notably expressives. \parencite[1]{williams_aesthetics_2014}
\end{quotation}

De tels phénomènes ont notamment été étudiés en détail en vietnamien \parencite{brunelleetal2014} et en japhug \parencite{jacques2013c}. Leur exploration en na, modestement entamée en 2024, est à un stade peu avancé: quinze interjections, huit idéophones. Cela constitue déjà un certain progrès par rapport aux débuts, et une direction dans laquelle continuer à avancer à l'avenir. L'idée (en apparence iconoclaste et excessive) d'un renversement de perspective, par lequel des éléments souvent considérés comme marginaux se trouveraient au centre de l'attention, présente un réel potentiel heuristique, en déplaçant les perspectives au sujet des mécanismes de la communication. C'est ainsi que Mark Dingemanse évoque, dans un travail au sujet des interjections, «{\dots}~a~potentially
radical reversal: from interjections at “the outskirts of real language” (Müller 1861) to interjections at the heart of language» \parencite[258]{dingemanse_interjections_2024}. Cette proposition amène à prendre du recul par rapport aux pratiques habituelles des linguistes, et à prêter l'oreille aux réflexions de collègues d'autres disciplines au sujet des relations entre signifiant et signifié.

\begin{quote}
« N'oublions pas qu'au départ on a, à tort, qualifié d'arbitraire le rapport du signifiant et du signifié. C'est ainsi que s'exprime, probablement contre son cœur, Saussure -- il pensait bien autre chose, et bien plus près du texte du \emph{Cratyle} comme le montre ce qu'il y a dans ses tiroirs, à savoir des histoires d'anagrammes. Or, ce qui passe pour de l'arbitraire, c'est que les effets de signifié ont l'air de n'avoir rien à faire avec ce qui les cause.
%Seulement, s'ils ont l'air de n'avoir rien à faire avec ce qui les cause, c'est parce qu'on s'attend à ce que ce qui les cause ait un certain rapport avec du réel. (…)
(…) Cela veut dire que les références, les choses que le signifiant sert à approcher, restent justement approximatives -- macroscopiques par exemple. Ce qui est important, ce n'est pas que ce soit imaginaire -- après tout, si le signifiant permettait de pointer l'image qu'il nous faut pour être heureux, ce serait très bien, mais ce n'est pas le cas. Ce qui caractérise, au niveau de la distinction signifiant/signifié, le rapport du signifié à ce qui est là comme tiers indispensable, à savoir le référent, c'est proprement que le signifié le rate. Le collimateur ne fonctionne pas. » (Jacques Lacan, \emph{Le Séminaire}, tome XX, Paris: Éditions du Seuil, 1975, p. 23)
\end{quote}

Une des ambitions du travail qui sous-tend l'entreprise documentaire (dont sont issus ce dictionnaire, l'ensemble de textes recueillis, et les travaux descriptifs et analytiques) est de conserver une distance critique vis-à-vis des définitions et procédures de la «linguisterie», afin de se rendre disponible pour prêter attention au glissement incessant du signifié sous le signifiant. Pour saisir une perche que nous tend Grothendieck:

\begin{quotation}
    Nul «dictionnaire» {\dots} ne peut se substituer à la qualité d'attention et de présence de celui «à l'écoute»! (Pas plus qu'un dictionnaire ne pourrait nous donner la clef d'une compréhension d'une seule parmi l'infinité des situations vécues qui forment la trame de notre vie.) \cite[1257]{grothendieck_recoltes_2021}
\end{quotation}

Il paraît possible, pour soustraire la lexicographie à une critique somme toute sans objet, de recomposer les éléments de ces deux affirmations qui l'une et l'autre vont de soi. Un dictionnaire n'a pas vocation à se substituer à quoi que ce soit; en revanche, il constitue un outil important pour aborder une langue rare, et peut, à défaut de clefs, fournir des indications tout à fait précieuses à qui se place à l'écoute des propos tenus dans cette langue.

En conclusion, soulignons que le dictionnaire comporte de nombreuses lacunes, ainsi que des approximations et, à n'en pas douter, des erreurs. Les lectrices et lecteurs sont encouragé·es à nous faire parvenir critiques et corrections.


\section{Remerciements}

En premier lieu, mes remerciements vont à Latami Dashilamu \pcmn{拉他咪打史拉姆}, locutrice de référence et co-autrice du présent travail, pour la motivation, la persévérance et la générosité avec lesquelles elle est engagée depuis 2006 dans notre collaboration. Mes remerciements chaleureux à son fils Latami Dashi \pcmn{拉他咪王勇(拉他咪达石)} pour avoir encouragé et soutenu mon travail avec sa mère, et pour avoir fait bénéficier le dictionnaire de ses connaissances expertes au sujet de la société na. Merci à tous les membres de la famille pour leur patience et leur bonne humeur au fil des années.

Mes remerciements particuliers à Benjamin Galliot, qui assure volet informatique et volet typographique depuis l'automne 2016. La philosophie de son travail est que l'outil informatique ne doit pas brider le bon développement du dictionnaire, mais au contraire l'accompagner et le favoriser. Son degré d'implication très exceptionnel permet à chaque dictionnaire d'être unique dans sa structure, son élaboration et sa mise en page, afin de s’adapter au mieux aux spécificités de la langue décrite et aux souhaits des lexicographes. C'est son travail qui a donné corps au dictionnaire à partir de la version 1.2, de la structure générale jusqu'aux plus petits détails typographiques. Son attachement à une esthétique soignée nous encourage à mettre les bouchées doubles afin que le contenu du dictionnaire ne soit pas trop en décalage, par ses imprécisions, lacunes et erreurs (inévitables dans l'exploration d'une langue à tradition orale), avec la beauté typographique de l'objet éditorial. La chaîne de traitement flexible et dynamique qu'il a mise en place, qui permet une mise à jour des fichiers XML et PDF à mesure de la progression de l'enquête, encourage également l’enrichissement de la base de données.

Vifs remerciements également à Séverine Guillaume et Céline Buret, qui ont planifié et mis en œuvre la mise en forme informatique du dictionnaire jusqu'à sa version 1.1. Merci à Mathieu Mangeot-Nagata et Laurent Romary pour leurs conseils experts en matière de lexicographie informatique; à Guillaume Jacques, qui montre l’exemple par un dictionnaire exceptionnel de la langue japhug; aux connaisseurs de la culture na pour nos échanges au fil des ans: Lamu Gatusa \pcmn{拉木·嘎吐萨} (nom de plume: Shi Gaofeng \pcmn{石高峰}), Liberty Lidz, Christine Mathieu, Maxime Fily.

Merci à Nathan Hill et Tsering Samdrup pour l'identification des noms d'origine tibétaine, et à Gerong Pincuo \pcmn{给汝品初} pour les comparanda en langue pumi. Merci à Yi Li \pcmn{衣莉} pour ses suggestions de corrections et ses conseils avisés. Merci à A Hui \pcmn{阿慧} pour sa relecture d’une partie du dictionnaire en 2014.

Mes remerciements particuliers à Roselle Dobbs (Ddeema Lhaco, \pcmn{杜玫瑰}) pour le flux régulier d’informations, de corrections et de conseils depuis les débuts de l’élaboration de ce dictionnaire, et pour avoir doté le dictionnaire d'informations orthographiques pour chaque entrée. \texteeng{``Blessed are the true fault-finders, for they shall be called midwives of truth''} \parencite[vi]{yliniemi_descriptive_2022}.

Les auteurs sont seuls responsables des erreurs et limitations qui demeurent.

Merci au Centre de recherches sur la culture dongba (\pcmn{丽江市东巴文化研究院}) à Lijiang, et à l’Université du Yunnan à Kunming, pour leur aide dans la gestion des questions administratives liées aux enquêtes de terrain.

J’ai reçu dans mon travail l’aide de tant de personnes que je crains de ne pouvoir rendre justice à toutes; que celles et ceux que j’oublie de nommer ici veuillent bien accepter mes excuses pour cette négligence.

Ce travail a été soutenu financièrement par le projet ANR «Himalayan Corpora» (HimalCo, ANR-12-CORP-0006) et par le Labex «Empirical Foundations of Linguistics» (EFL, ANR-10-LABX-0083). Il a en outre bénéficié des résultats des projets ANR «Phylogenetic assessment of Southern Qiangic» (PASQi, ANR-07-JCJC-0063), «Computational Language Documentation by 2025» (CLD2025, ANR-19-CE38-0015), «Probing neural representations for typological signal» (DeepTypo, ANR-23-CE38-0003) et «Glottalization in the light of Machine Learning» (Glot-TAL, ANR-24-CE38-3766).

Je dédie ce travail à Juliette Zhao \pcmn{赵筱筠}, ma femme bien-aimée.

{\raggedleft Alexis Michaud\par}
