\languefra

\chapter*{Preface}

Language is a fundamental dimension of a people's culture. It is the vehicle of knowledge; it also reflects the history through which society shapes and reinvents itself. Experts and academics around the world are fully aware of the seriousness of the phenomenon whereby a growing number of languages are threatened or even disappearing. Linguistic diversity, an essential manifestation of human cultural diversity, becomes eroded under our eyes.

The Moso people have their own language and culture, but the continuing development of cultural tourism in the Moso regions is leading to a major crisis in the transmission of the Moso mother tongue. It is in this unprecedented context that Alexis Michaud, a researcher at the LACITO (\em{Langues et Civilisations à Tradition Orale}) research centre within \emph{Centre National de la Recherche Scientifique}, my dear mother, Mrs. Latami Dashilamu, and Pascale-Marie Milan, a doctor in anthropology, have co-written their \emph{Dictionary of the Moso dialect of the village of Alawua}. The fruit of eighteen years' effort, this work, published in the CNRS “Lexica” series of dictionaries, makes a major contribution to the protection and development of the Moso language. It provides an example of successful action in the space of the conservation and transmission of endangered minority languages.

In my threefold capacity as a witness to the history of this work, as a participant in the enterprise, and as the book's editorial consultant, I take the opportunity of this preface to say a word about my personal feelings and experiences, at this historic moment when the \emph{Dictionary of the Moso dialect of the village of Alawua} is being published.

\section*{A look at designations of the Moso: endonyms and exonyms}

The Moso call themselves ‘Na’. Their name comes in a wide variety of phonetic transcriptions in Chinese (such as ‘Nari’, ‘Nahan’, ‘Naheng’, ‘Naru’ and ‘Naruo’). The term ‘Na’ has the meanings of ‘great’, ‘magnificent’ and ‘black, dark’. The term ‘Moso’ is an exonym whose earliest attestations date back to the Qin-Han period (from 221 BCE to 220 CE), and which has various variants. Western academic works use a variety of terms. Some aim to transcribe the endonym -- either directly or through the prism of Chinese transcriptions -- resulting in ‘Na’, ‘Nari’, ‘Naru’, ‘Naze’ and ‘Nahing’. Others take as their starting point the Chinese exonym \pcmn{摩梭}, adapting it where necessary to the pronunciation habits of the target language, as ‘Mosuo’, ‘Mosso’, or ‘Moso’.

\section*{A look back at the years of our acquaintance}

It was eighteen years ago that I first met Alexis Michaud, with whom I've been working ever since. A newly qualified doctor in linguistics, he was gifted, wise and determined. Alexis Michaud became passionately attached to the Moso culture. He developed an emotional bond with the language as he became part of our Moso family. Alexis Michaud's family has developed a deep bond with ours: his wife, Juliette Zhao, who has a wisdom informed by both Eastern and Western cultures, and his daughter Alice, born the same year as my eldest daughter, Dashiduma. His parents and brother, as well as the parents of his beloved Juliette Zhao, visited us. We are so familiar with each other that even the people who know us best were astonished at how easily we understood each other. It was on this basis that the dictionary project was carried out. Above all, the success of the project is due to the intellectual and human qualities of my mother Latami Dashilamu. Loving, warm, kind, patient, persevering and always smiling, she was Alexis Michaud's mentor in his research into the Moso language, constantly lavishing unparalleled affection and attention.

\section*{A look at Alexis Michaud's research into the Moso language}

In 2010, A. Michaud and I co-authored a paper that was presented at the 16th Congress of the International Union of Anthropological and Ethnological Sciences (IUAES) and published in Chinese in the journal \texteeng{Lijiang Ethnic Studies}. It was published in English under the title \texteeng{“A description of endangered phonemic oppositions in Mosuo (Yongning Na)”}, in the book \texteeng{Issues of language endangerment} edited by Tjeerd De Graaf, Xu Shixuan and Cecilia Brassett. In 2017, the German publisher Language Science Press published a 573-page monograph by Alexis Michaud: \emph{\texteeng{Tone in Yongning Na: lexical tones and morphotonology}}. James A. Matisoff, a professor at the University of California at Berkeley and an internationally renowned expert on Sino-Tibetan languages, described the book as a monumental work in the study of tones, setting a benchmark for all future studies on the morpho-tonology of the Tibeto-Burman area. Professor Lee Wai-sum of the City University of Hong Kong called it a professional, comprehensive and meticulous work, a true classic of academic research. Professor Juliette Blevins of the City University of New York said the book provided a remarkable analysis of tonal categories, phonological patterns of tone-syllable combinations, and morpho-tonology in various grammatical structures. Not only does the book provide new content, it also has a clear and engaging narrative style, which does not simply describe linguistic structure in terms of pre-established models, but gives the reader the opportunity to weigh the relevance of the analytical choices proposed. The book can be used as a tool for learning how to analyze primary linguistic data, and as a manual for linguistic documentation and language description.

\section*{A look at the dictionary of the Moso dialect of Alawua}

When Alexis Michaud first visited the village of Alawua, in Yongning, to study the Moso language, his initial intention was to confine himself to a phonological analysis. He was surprised to discover that Moso tones were not only a purely phonetic-phonological issue, but were also inextricably linked to grammar. This discovery prompted him to undertake a comprehensive study of Moso. His research methodology is based on the collection of a sizeable corpus: vocabulary from texts feeds this dictionary. Compiling the dictionary is a lengthy process, based on traditional fieldwork methods.

In fieldwork, the transcription and analysis of the corpus is a complex, systematic process. Alexis Michaud considers phonetic transcription and corpus analysis to be crucial and interdependent steps, on which the scientific validity and reliability of the results depend: for a dictionary, what is at stake is the authenticity of the lexicon. The methodology deployed involves the use of the International Phonetic Alphabet (IPA). Recently, the transcription of the Alawua Moso dialect has benefited from state-of-the-art software tools for automatic speech recognition, fine-tuned specifically on the basis of recordings from the Alawua dialect. Speech is clearly recorded using high-fidelity audio equipment, then transcribed into written text word by word, preserving the corpus in its original form, including non-linguistic features such as repetitions, pauses and intonation.

In the process of corpus analysis, Alexis Michaud pays particular attention to morphological analysis, frequency of word occurrence, and lexical distribution. He analyzes the grammatical rules and syntactic features of Moso based on the structures of sentences found in the corpus. He sets about analyzing the corpus from several angles, starting with phonology, vocabulary and grammar, in order to reveal the laws and intrinsic characteristics of the Moso language.

\section*{Perspectives concerning the future of research on Moso language and culture}

This dictionary of the Moso language of the village of Alawua not only records the language and the cultural traditions of the Moso people, it also reflects the changes and evolution of the Moso people over the course of history. Through the explanations provided in the dictionary, insights can be gained into the evolutionary trajectory of Moso society, and the variegated facets of the Moso cultural heritage. This dictionary of a Moso dialect bears witness to the essence and soul of Moso culture. The publication of the \emph{Dictionary of the Moso dialect of the village of Alawua} opens up access to precious spiritual riches. They can be passed on to future generations, to accompany them on their journey.

The process of compiling the dictionary also has the effect of bringing about a form of standardization of the Moso language, which can help progress in research in linguistics, ethnology and history, while facilitating dialogue between disciplines. Finally, the dictionary's explanations of words and example sentences also constitute valuable references for learners.

In conclusion, the \emph{Dictionary of the Moso dialect of the village of Alawua} is a fine achievement, which promotes a better understanding of the language and culture of the Moso people on the part of academics, and provides a solid basis for in-depth research in various disciplines. It is hoped that this dictionary will help to improve understanding and communication between the different ethnic groups, to the greater benefit of national unity and social harmony. This dictionary of a Moso dialect can play an important role in the field of Moso cultural heritage, as well as being an important educational resource in the regions where the Moso people live. The work helps to strengthen the scientific and cultural knowledge of the Moso people, and can inspire legitimate pride among speakers. The Chinese nation is a large family made up of 56 nationalities, each with its own language, culture and historical tradition. The publication of the Moso dialect dictionary is a precious contribution to the body of knowledge about the world's languages and cultures.

{\raggedleft Latami Wangyong \pcmn{拉他咪王勇}\par}

%{\raggedleft 2024年初冬时节\par}
