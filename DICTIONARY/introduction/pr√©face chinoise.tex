\languecmncn

\chapter*{序言}

语言是世界上各民族文化的重要组成部分,承载了各民族的历史、社会、文化及传统知识等关键信息。语言的多样性是文化多样性的突出体现。近年来,越来越多的少数民族语言面临濒危甚至消失,这一现象已引起世界各国专家学者的广泛关注。

摩梭人拥有独特的语言和文化。随着摩梭地区文化旅游事业的不断升级与发展,摩梭母语的保护与传承面临巨大危机。鉴于上述现状,法国国家科学研究院世界语言与口传文化研究所研究员、博士生导师\textefra{Alexis Michaud}(一史棣嘟)博士、我亲爱的母亲拉他咪打史拉姆先生和法国人类学博士\textefra{Pascale-Marie Milan}(瑸妈拉姆)合著的《永宁阿拉瓦村摩梭方言-汉-英-法词典》,经过18年不懈努力,终于在法国国家科学研究院的Lexica词典系列中出版。他们为摩梭语言的传承、保护与发展作出了杰出贡献,亦为全球面临濒危少数民族语言的保护、传承实践提供了一个成功的范例。

作为这一成果的见证者、参与者和审校者,我在《永宁阿拉瓦村摩梭方言-汉-英-法词典》出版的这一历史性时刻,借此机会撰写序言,并谈谈个人的感受与体会。

\section*{穿过记载摩梭称谓的历史典籍之眼}

摩梭人在达巴口诵经典中自称“纳”“纳日”“纳罕”“纳哼”“纳汝”“纳如”“纳恒”等;“纳”的意思是“大”“广”“伟大”“宏伟”“浩瀚”“黑森森”等,“日”“罕”“哼”“汝”“如”“恒”均指人。摩梭最早的他称见诸于秦汉时期《史记·西南夷列传》《汉书》《后汉书》,记载为“牦牛夷”。晋代常璩撰写的《华阳国志》记载为“摩沙”;唐代樊绰的《蛮书》记载为“麽些”;元代记载为“摩娑”;清代称“麽些”。民国时期,陶云逵在《麽些象形文字字典》中记录为“麽些”,西方学术著作中则称之为 \texteeng{Moso}、\texteeng{Mosuo}、\texteeng{Naze}、\texteeng{Na},前两者为音译他称,后两者为自称的音译。

\section*{穿过我们相识的岁月之眼}

2006年10月,我与\textefra{Alexis Michaud}博士——中文名米可,摩梭名一史棣嘟——在多方努力下见面并开始我们的合作研究。\textefra{Michaud}博士是一位天资聪颖、睿智且坚毅的语言学家,18年来,他深深地热爱摩梭文化,融入我们的摩梭家庭,与摩梭语产生深厚的情感。他的中国妻子赵筱筠智慧且精通于东西方文化,女儿爱丽思与我的大女儿达诗笃玛同一年出生,\textefra{Alexis Michaud}博士全家和我们家有着很深厚的感情,他的父母亲和弟弟、及他爱人赵筱筠的父母都访问过丽江,我们都熟悉各自的家庭成员,以至于我的小女儿慈琳拉姆出生后,第一次与\textefra{Alexis Michaud}博士见面时,令我爱人很诧异于那份熟悉、温暖的氛围。这部词典的诞生与成功,与我慈爱、温暖、善良、忍耐、坚持和脸上总是挂着慧笑的母亲拉他咪打史拉姆密不可分,她是\textefra{Michaud}博士做摩梭语言研究的导师,以及生活上给予无与伦比的关爱和照顾的阿妈。

\section*{穿过\textefra{Alexis Michaud}博士做摩梭语言研究的成果之眼}

\textefra{Michaud}博士做研究很勤奋、很严谨并取得丰硕的成果,在2010年\textefra{Michaud}博士与我合作发表的《云南省丽江市永宁区域摩梭话中濒危消失的声调和音位》,载于云南民族出版社出版的《丽江民族研究》。2012年出版的国际人类学联合会第十六届大会文集,由[中]徐世璇、[荷]郭天德和[英]廖乔婧主编的《濒危语言\texteeng{Issues of Language Endangerment}》中收入\textefra{Michaud}博士与我合作的《\texteeng{A Description of Endangered Phonemic Oppositions in Mosuo (Yongning Na)}》。2017年德国语言科学出版社出版了\textefra{Alexis Michaud}博士573页的英文专著《\texteeng{Tone in Yongning Na: Lexical Tones and Morphotonology}(永宁摩梭语的声调:词汇调与形态调系规则)》,加州柏克莱大学教授、国际知名汉藏语系语言研究专家\texteeng{James A. Matisoff}评价这是一本象征着藏缅语支声调研究最高水准的著作,它有着极其详实的细节描写及深刻的科学分析,这本声调研究的鸿篇巨制为未来所有的藏缅声调形态研究树立了一个标杆。香港城市大学李惠心教授评价这是一本专业著作,全面及细致地去描绘汉语语系或藏缅语系中请言的声调和声调模式,这本书绝对是学术界的经典著作。纽约市立大学研究生院\texteeng{Juliette Blevins}教授评价本书提供了一个恢弘的声调分析,声调类别、声调与音节结合的音系规律、以及备种语境下语法结构中特有的变调规律。其中不仅给出了全新的内容,也具备独特的叙事风格,不是从既有视角来复述语言结构,更给读者以一个进阶式的分析途径;此著作具备了广泛的语言学应用基础,一部对第一手研究语料的使用示范,一部语言田野调查的基础教材,一部语言记载教程,一部语言描写教程。

\section*{穿过摩梭方言词典编纂之眼}

\textefra{Alexis Michaud}博士刚到永宁镇(当时行政区划称“永宁乡”)阿拉瓦村调查摩梭话时,初衷只是想进行音系分析。经过一段时间普查,他惊奇地发现摩梭语声调不仅是单纯音系问题,同时与语法有着千丝万缕的联系。这一发现促使他开始对摩梭话进行全方位研究。研究方法以搜集到的长篇语料为蓝本,在搜集语料的同时,以长篇语料中的词汇为基准,汇集成本部词典。词典编撰本身是一个漫长过程,采用传统田野调查方法。

在田野调查过程中,记音、转写语料和语料分析是一个系统而复杂的过程。只有做好每一个环节,才能确保研究成果的科学性、可靠性和词典编纂的真实性。\textefra{Alexis Michaud}博士将记音、转写语料及语料分析视为至关重要且相互关联的步骤。当时的记音方法包括国际音标以及他专为摩梭语方言研发的记音方案。借助高质量录音设备,清晰录制了发音,并按照录音逐字逐句地将其转写为书面文字,完整保留了语料原貌,包括重复、停顿、语气词等非语言特征。

在语料分析过程中,\textefra{Alexis Michaud}博士着力分析词汇构成、频率和分布等特点。根据语料中的句子结构,他分析了摩梭语语法规则和句法特征,从词汇、语法、语音等多个维度对语料进行了全面分析,以揭示摩梭语内在规律和特点。

\section*{穿过未来摩梭语言文化研究之眼}

《永宁阿拉瓦村摩梭方言-汉-英-法词典》不仅记录了摩梭人的语言文字和文化传统,还反映了摩梭人在历史长河中的变迁和发展。通过词典中的词条和解释,可以看到摩梭人社会历史的演变轨迹和文化传承的脉络。摩梭方言词典中蕴含着丰富的民族精神和价值观念,这些精神和价值观念是摩梭文化的精髓与灵魂。

通过编纂和出版词典,可以将这些宝贵的精神财富传承下去,激励后人不断追求卓越、奋发向前。《永宁阿拉瓦村摩梭方言-汉-英-法词典》是记录和传播摩梭人语言文化的重要载体,通过方言词典,可以系统地保存和传承摩梭人的语言、文字、文化及历史传统,为后人研究提供宝贵的资料。词典的编纂过程本身就是对摩梭民族语言的一次深入研究和规范,有助于推动语言学、民族学、历史学等相关学科的发展。同时,词典中的词条解释、例句等也为语言学习者提供了重要的参考。

《永宁阿拉瓦村摩梭方言-汉-英-法词典》的出版是学术成果的重要体现,丰富了国内外学术界对摩梭人语言文化的认识,也为相关领域的深入研究提供了翔实的资料基础。通过编纂和出版摩梭方言词典,可以增进不同民族之间的理解和交流,促进民族团结与社会和谐。摩梭方言词典在摩梭文化传承领域发挥着重要作用,为摩梭人生活地区的文化教育提供了重要教材和资源,有助于提升摩梭人群众科学文化素养和民族自豪感。中华民族是56个民族共同组成的大家庭,每个民族都有其独特语言文化和历史传统,促进各民族交往、交流、交融具有重要现实意义。摩梭人方言词典的编纂与出版,丰富了世界语言文化的内涵与外延,为世界语言文化增添了宝贵财富。

是为序。

{\raggedleft 拉他咪王勇\par}

{\raggedleft 2024年初冬时节\par}
