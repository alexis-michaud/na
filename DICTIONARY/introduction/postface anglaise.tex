The Na-Chinese-French-English dictionary is the third dictionary in the Lexica collection. It is the last in the set of Lexica dictionaries whose earliest versions predate the current Lexika software suite and full-fledged Lexica collection (the version of the Na dictionary generated by the old PLexika software was numbered version 1.2).

The first distinctive feature of this dictionary is that it comes in three distinct versions, depending on the target language: Chinese, English and French. They are not symmetrical, the Chinese being the version that can be considered exhaustive (including all languages and notes), the English and French being mutually expunged of the other language. It was technically interesting to try and factorize to the greatest possible extent the code and templates specific to the three versions, in order to have a robust and flexible means of correcting and improving the three versions simultaneously, while retaining their specific features.

A second feature of this dictionary (or set of dictionaries) is a technical one: the use of metadata tags in the source \textsc{Lex} file. At first glance, this seems to make the structure more cumbersome, but it is in fact necessary due to the multiplication of diverse languages and of auxiliary metadata. It would obviously have been easy to convert these metadata into as many (sub)tags as necessary in order to keep a format that would resemble the original \textsc{Mdf} format, but I felt that an intermediate structural level between the tag and the data properly speaking was fairly relevant. Accordingly, this feature was integrated into the main algorithm of the (J)Lexika engine. The ensuing downstream difficulties (notably matters of over-definition and priority between these metadata and those inherent in the tag – or in the absence thereof) were worth overcoming.

Another special feature of the dictionary is an aesthetic one: the addition of letterheads in the page footer, to provide the reader with a clear overview of the various graphemes used in the language under study. This footer sets out the chosen lexicographical order. It also teases apart vowels and consonants. Boldface singles out the letterhead corresponding with the current page's location, and all other letterheads have links allowing smooth access to the relevant sections. This feature will certainly be added to the other dictionaries in the collection.

It is also worth noting that particular wishes with respect to the layout of the Chinese cover offered an opportunity for me to look into what \LaTeX{} has to offer in this space. The results were highly satisfactory: aesthetic considerations did not conflict with programmatic flexibility. This paved the way. This led me to decide to pave the way for freer and less sober covers than initially envisaged for all dictionaries in the series. After all, as I write these lines, the first two dictionaries in the collection, which have been delayed for various reasons, still don't have a finished cover, so this rearrangement of the timetable will have been, against all expectations, beneficial for these two latecomers!

A final specificity of this dictionary project is the highly sustained interaction between the linguist, \alexisfra, and myself. A lot of thought and discussion has taken place through various channels, including face-to-face interaction, ’phone/video conversations, e-mail, Zulip, and the Github software forge, leading to numerous changes to the structure of the source data and improvements to the introduction, including some changes that I carried out directly myself. Many of these changes and discussions are saved in the forge tickets, which will serve as a valuable, open historical record.

The elaboration of the Chinese version was also an opportunity to discuss certain Chinese translations, in particular the name of the collection itself. More generally, this dictionary has also provided an opportunity to reflect on the notion of authorship in a thoughtful and conscientious way. The aim is to give due recognition to all those who have contributed, each in their own way, to the lexicography project.

Finally, the hardest part for me is knowing that, despite all the care taken, there shall still be typos and imperfections in the released version, as is inevitable in any work of this type and size. Generally, I can live with this, as the Lexika processing chain is designed to generate a brand new dictionary very easily as soon as there is an update to the data or an improvement to the technical architecture (stylistic changes, new finishing touches, and such). But in this particular case, a print run is considered at the end, so it's not just a digital dictionary that could exist in a dozen versions! No: if you have a paper version in your hands, perhaps it will stay for decades (which is what we all hope!) and the errors shall be engraved in the fibres of the paper, year after year. As a perfectionist, it's very hard for me to accept that a document is printed when it could still be further polished and perfected. But we all know the adage about perfection (\textefra{La perfection n'est pas de ce monde}; an English saying even has it that “Perfect is the enemy of good”), so we have to accept it.

In conclusion, I hope that this book will benefit the Na people, and that it will contribute to the preservation and transmission of their rich cultural heritage.

\bigskip

\hfill \benjaminfra
