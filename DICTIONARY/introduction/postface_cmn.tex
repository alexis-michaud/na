《永宁阿拉瓦村摩梭方言—汉—英—法词典》是“世界珍稀语言系列词典丛书”中的第三部。这是2.0版本(之前曾有由旧\textefra{PLexica}软件生成过的1.2版本)。

这本著作的一个特点在于它的三种语言版本:中文、英文和法文。这三个版本不完全对称。中文版堪称最完整版本,包括所有语言,甚至注释信息都带有三种语言翻译,详尽无遗。英文版和法文版中的外文翻译则相互删节:英文版不带法语翻译、法文版不带英语翻译。从技术角度讲,将三个版本的代码和模板尽可能地因素化,这是非常有趣的过程。这样既可以有一个强大而灵活的手段来同时修正和改进三个版本,又可以保留它们各自的特点。

除了有中、英、法这一重要语言特点外,这部词典还有一个技术方面的特点。它的源文件(\textefra{\textsc{Lex}}文件)中使用元数据,乍一看似乎会使结构更加繁琐,但实际上这种做法是合适的,由于语言和其他辅助元数据的增多,将元数据包含在源文件里,是很有必要的。显然,为了保持与原始\textefra{\textsc{Mdf}}文件格式相似的格式,将这些元数据转换成尽可能多的(子)标签是很容易的,但我认为在标签和数据之间再插入中间结构层次是合理的,能起重要作用,至于我决定将这一功能集成到了\textefra{(J)Lexika}软件的核心算法引擎中。这个重要改进导致了不一般的困难,特别是元数据与标签信息之间的“过度定义”(\texteeng{\emph{overdefinition}})和优先级判断。但一个一个克服这些困难,最终是完全值得的,是提高\textefra{(J)Lexika}软件的水平的重要机会。

在本词典处理排版过程中给“世界珍稀语言系列词典丛书”带来的另一个创新是在页面的页脚增加了信头,偶数页(左页)显示摩梭语所有声母、奇数页(右页)显示摩梭语所有韵母。这样,就为读者提供了该语言中使用的不同词素的清晰概览,显示所选择的词法顺序与元音和辅音之间的分隔。还用粗体显示本页面所属信头。每个信头还包含链接,使得词典更容易浏览。该系列的其他词典也将会增加这一功能。

本词典的排版工作在封面设计方面也给“世界珍稀语言系列词典丛书”带来了重要启发。我借设计中文封面的机会了解了\LaTeX{}在这方面的功能。结果非常令我满意:美观方面的考虑并不矛盾于程序的灵活性。因此我决定为系列所有的词典的封面铺平道路,让封面设计比最初设想的更自由、更丰富。毕竟,就在我写下这些文字的时候,世界珍稀语言系列的第一批词典(第一、第二本词典)由于种种原因被推迟了还未问世,因此,第三本先正式问世,不出所料地为即将出版的词典带来相当重要的好处!

在我眼中,本词典还有另一个特点,就是语言学家(米可)和我本人之间非常持续的互动。通过各种渠道,包括面谈、电话、视频、邮件、\textefra{Zulip}、\textefra{Github}等等,我们进行了大量的思考和讨论,从而对源数据的结构进行了大量修改,并对前言进行了改进,有时甚至是我本人直接进行的改进。其中许多修改和讨论公开保存在\textefra{Github}锻造票据,这些票据将成为宝贵的历史记录。

中文版也是讨论某些中文翻译的机会,特别是词典系列的名称。从更广泛的意义上讲,这本词典出版的过程中也提供了一个机会,让我们深思熟虑“著者”这个概念。目的在于认真负责给对所有以各自的方式为词典编纂工程做出贡献的人们给予应有的认可。

最后,对我来说,很难受的一个事实在于,尽管已经尽了努力,谨慎措施,但与任何此类规模的作品一样,《永宁阿拉瓦村摩梭方言—汉—英—法词典》仍然会有错别字和不完善之处。一般来说,我可以接受这一事实。而且\textefra{Lexika}处理链的设计刚好允许随时更新后立,内容或表面处理改善后轻松生成一本全新的词典版本。但本词典最后还要印刷,它不仅仅是一本可以有好几个版本的数码词典。假如您手中拿着的是纸质版本,也许它将保留几十年(这正是我们所希望的!),而其错误刻在纸张的纤维中!作为一个完美主义者,我很难接受在文件还可以完善的情况下就将其打印出来。但我们都知道“世界没有十全十美的事情”这句格言,所以我不得不接受。

最后,我希望这本书能够造福摩梭人民,为保护和传承摩梭人丰富的文化遗产做出贡献。

\bigskip

\hfill \benjaminfra

