Le dictionnaire na-chinois-français-anglais est le troisième dictionnaire de la collection Lexica, ainsi que le dernier qui avait préexisté dans une version générée par l’ancien logiciel PLexika (la version en question était numérotée 1.2).

Sa première particularité est d’être en trois versions distinctes selon la langue cible : le chinois, le français et l’anglais. Elles ne sont pas symétriques, la chinoise étant la version pouvant être considérée comme exhaustive (comprenant toutes les langues et les notes), les française et anglaise étant mutuellement expurgées de l’autre langue. Il était techniquement intéressant de chercher à factoriser au maximum le code et les gabarits propres aux trois versions afin d’avoir à la fois un moyen robuste et flexible pour corriger et améliorer les trois versions simultanément, tout en gardant leurs spécificités.

Une autre particularité de ces dictionnaires, technique cette fois, est l’utilisation de métadonnées dans le fichier \textsc{Lex} source, ce qui semble de prime abord alourdir la structure, mais qui est en réalité nécessaire du fait de la multiplication des langues et d’autres métadonnées auxiliaires. Il aurait été évidemment aisé de convertir ces métadonnées en autant de (sous-)balises nécessaires pour garder un format ayant l’allure du format \textsc{Mdf} original, mais un niveau structurel intermédiaire s’intercalant entre la balise et la donnée \emph{stricto sensu} m’a paru plutôt pertinent, au point d’avoir intégré cette particularité dans l’algorithme principal du moteur (J)Lexika ; les difficultés en aval (notamment la gestion des conflits entre ces métadonnées et celles liées à la balise, ou leur absence pure et simple) valaient la peine d’être surmontées.

Une autre particularité, esthétique cette fois, est l’ajout des lettrines dans le pied de page, afin de proposer au lecteur une vue synthétique et claire des différents graphèmes utilisés dans la langue étudiée, avec l’ordre lexicographique choisi, la séparation entre voyelles et consonnes, un marqueur clair montrant à quelle lettrine la page se situe, ainsi que des liens permettant d’accéder aux lettrines voulues, le cas échéant. Cette fonctionnalité sera très certainement ajoutée aux autres dictionnaires de la collection.

Notons aussi que du fait de souhaits particuliers quant à l’allure de la couverture chinoise, j’en ai profité pour voir ce qu’il était possible de faire en \LaTeX{} à ce niveau-là. Les résultats étaient très satisfaisants, l’esthétique n’ayant pas limité la flexibilité programmatique, aussi ai-je souhaité à mon tour ouvrir la voie à des couvertures plus libres et moins sobres qu’initialement envisagé, et ce, pour tous les dictionnaires. Après tout, à l’heure où j’écris ces lignes, les deux premiers dictionnaires de la collection, qui ont eu des retards pour des raisons variées, n’ont pas encore de couverture terminée, ainsi ce réagencement d’emploi du temps aura été, contre toute attente, profitable pour les retardataires !

Une dernière particularité de ce projet de dictionnaires est l’interaction très soutenue entre le linguiste, \alexisfra, et moi-même. Beaucoup de réflexions et de discussions ont eu lieu par divers canaux : échanges de vive voix, par téléphone, par visioconférences, sur la messagerie Zulip et la forge logicielle Github, menant à de nombreuses modifications de la structure des données sources et l’amélioration des textes liminaires, parfois directement par moi-même dans le dépôt logiciel. Nombre de ces changements et discussions sont sauvegardés dans les tickets de la forge, qui servira ainsi d’historique précieux.

Par ailleurs, la version chinoise a été l’occasion de poser sur la table les réflexions autour de certaines traductions chinoises, notamment celle du nom de la collection elle-même. De manière plus générale, ce dictionnaire a aussi permis de s’interroger de manière réfléchie et consciencieuse sur la notion et la qualité d’\emph{auteur}, et ce, afin d’apporter aux diverses personnes qui contribuent, chacune à sa façon, à l'entreprise de création de dictionnaires, la reconnaissance qu'elles méritent.

Finalement, le plus dur pour moi est de savoir que, malgré tout le soin apporté, comme tout ouvrage de ce type et de cette taille, il restera des coquilles, des imperfections ! Généralement, je peux m’en accommoder, car la chaîne de traitement Lexika est faite pour justement générer très facilement un tout nouveau dictionnaire dès qu’il y a une mise à jour des données ou de l’architecture technique (améliorant le style, les finitions, etc.), mais dans ce cas précis, comme il y a une impression à réaliser à la fin, il ne s’agit pas uniquement d’un dictionnaire numérique qui pourrait exister en une dizaine de versions ! Non, si vous avez une version papier entre les mains, peut-être qu’elle restera des décennies (c’est ce que nous espérons tous !) avec ses fautes gravées dans les fibres du papier, vieillissant avec lui ! Étant perfectionniste, c’est très dur pour moi d’accepter qu’un document soit imprimé alors qu’il reste perfectible, mais l’on connaît tous l’adage à propos de la perfection, donc il faut l’accepter.

Pour conclure, j'espère que cet ouvrage bénéficiera au peuple na, et qu'il contribuera à la préservation ainsi qu'à la transmission de leur riche culture patrimoniale.

\bigskip

\hfill \benjaminfra
