\languefra

\chapter*{Préface}

{\raggedleft \em{Traduit du chinois par Alexis Michaud}\par}


La langue constitue une dimension fondamentale de la culture d’un peuple. Véhicule des connaissances, elle reflète en outre l’histoire au fil de laquelle la société se façonne et se réinvente. Experts et universitaires du monde entier mesurent pleinement la gravité du phénomène par lequel un nombre croissant de langues sont menacées, voire disparaissent. La diversité linguistique, expression essentielle de la diversité culturelle humaine, s'érode sous nos yeux.

Le peuple moso possède une langue et une culture qui lui sont propres. Mais le développement continu du tourisme culturel dans les régions moso mène à une crise majeure pour la transmission de la langue maternelle moso. C’est dans ce contexte inédit qu’Alexis Michaud, chercheur au laboratoire de Langues et Civilisations à Tradition Orale du Centre National de la Recherche Scientifique, ma chère mère, Mme Latami Dashilamu, et Pascale-Marie Milan, docteur en anthropologie, ont coécrit leur \emph{Dictionnaire du dialecte moso du village d'Alawua, na – chinois – anglais – français}. Fruit de dix-huit ans d’efforts, cet ouvrage, publié dans la série de dictionnaires «Lexica» du CNRS, apporte une contribution de premier plan à la protection et au développement de la langue moso. Il fournit un exemple de réussite en matière de conservation et de transmission des langues minoritaires.

En ma triple qualité de témoin de l’historique de ce travail, de participant à l’entreprise, et de consultant éditorial de l’ouvrage, je saisis, en ce moment historique où le \emph{Dictionnaire du dialecte moso du village d'Alawua, na – chinois – anglais – français} est publié, l’occasion de cette préface pour dire un mot de mes sentiments et expériences personnels.

\section*{Un regard sur les désignations des Moso: endonyme et exonymes}

Les Moso se nomment eux-mêmes «Na», nom qui se décline sous une grande variété de transcriptions phonétiques en chinois (telles que «Nari», «Nahan», «Naheng», «Naru», ou encore «Naruo»). Le terme «Na» signifie «grand», «magnifique», et peut aussi désigner une couleur: «noir, sombre». Le terme «Moso» est un exonyme dont les premières attestations remontent à la période Qin-Han (de 221 avant l’ère commune jusqu’en 220 de l’ère commune), et qui connaît diverses variantes d’écriture. Les ouvrages universitaires occidentaux emploient divers termes. Certains visent à transcrire l’endonyme – soit directement, soit au prisme des transcriptions chinoises –, ce qui donne des «Na», «Nari», «Naru», «Naze», «Nahing». D’autres prennent pour point de départ l’exonyme chinois \pcmn{摩梭} «Mosuo», l’adaptant le cas échéant aux habitudes orthographiques de la langue de destination, ce qui donne «Moso» ou encore, en français, «Mosso».


\section*{Un regard sur les années de notre connaissance}

Le Dictionnaire du dialecte mosuo du village d'Alawa, publié dans la série des dictionnaires Lexica du CNRS, s'appuie sur une collecte de données réalisée essentiellement auprès de locuteurs de différents groupes d'âge du hameau d'Alawa (rattaché au village de Pingjing, à Yongning). Notre famille vit dans ce hameau depuis cinq générations. Le projet dans son ensemble, du début à la fin, doit sa réussite aux qualités intellectuelles et humaines de ma mère Latami Dashilamu. Aimante, chaleureuse, gentille, patiente, persévérante et toujours souriante, elle a été le mentor d’Alexis Michaud dans ses recherches sur la langue moso, lui prodiguant avec constance une affection et une attention sans pareils.

C'est en octobre 2006 que j'ai rencontré Alexis Michaud, avec qui s’est nouée une collaboration qui dure depuis lors. Fraîchement émoulu docteur, il était doué, réfléchi, et déterminé. Alexis Michaud s’est attaché passionnément à la langue et à la culture moso. Il a développé un lien émotionnel avec la langue en s'intégrant à notre famille moso. La famille d’Alexis Michaud a acquis un lien profond avec la nôtre: sa femme, Juliette Zhao, qui possède une sagesse nourrie des cultures orientale et occidentale, et sa fille Alice, née la même année que ma fille aînée, Dashiduma. Ses parents et son frère, ainsi que les parents de sa bien-aimée Juliette Zhao, nous ont rendu visite. Lors de sa première rencontre avec ma fille cadette Cilin Lamu, la familiarité et l'atmosphère chaleureuse de leur échange étaient telles qu'elles ont suscité la surprise de mon épouse.

\section*{Un regard sur les recherches d’Alexis Michaud au sujet de la langue moso}

En 2010, A. Michaud et moi-même avons coécrit un travail qui a été présenté au XVI\textsuperscript{e} Congrès mondial de l'Union internationale des sciences anthropologiques et ethnologiques (IUAS) et publié en chinois dans la revue \texteeng{Lijiang Ethnic Studies}. Il est paru en anglais sous le titre \texteeng{“A description of endangered phonemic oppositions in Mosuo (Yongning Na)”}, dans l’ouvrage \emph{\texteeng{Issues of language endangerment}} dirigé par Tjeerd De Graaf, Xu Shixuan et Cecilia Brassett. En 2017, la maison d'édition allemande Language Science Press a publié une monographie de 573 pages d’Alexis Michaud: \emph{\texteeng{Tone in Yongning Na: lexical tones and morphotonology}}. James A. Matisoff, professeur à l'université de Berkeley, expert de renommée internationale au sujet des langues sino-tibétaines, a décrit le livre comme un travail monumental dans l'étude des tons, qui établit une référence pour toutes les études à venir sur la morpho-tonologie de l’aire tibéto-birmane. Le professeur Lee Wai-sum, de la \emph{City University of Hong Kong}, a déclaré qu'il s'agissait d'un ouvrage professionnel, exhaustif et méticuleux, un véritable classique en matière de recherche universitaire. Le professeur Juliette Blevins de la \emph{City University of New York} a déclaré que ce livre fournissait une magnifique analyse des catégories tonales, des modèles phonologiques de la combinaison ton-syllabe, et de la morpho-tonologie dans les diverses structures grammaticales. L’ouvrage ne se limite pas à un apport de contenu nouveau. Il possède également un style narratif clair et engageant. Il ne se contente pas de décrire la structure linguistique à l’aune de modèles pré-établis, mais offre au lecteur l’occasion de mesurer la pertinence des choix analytiques proposés. L'ouvrage peut constituer un outil d'apprentissage concernant la méthode d’analyse des données linguistiques primaires; il a valeur de manuel pour la documentation linguistique et la description des langues.

\section*{Un regard sur le dictionnaire dialectal moso}

Lorsqu’Alexis Michaud s'est pour la première fois rendu dans le village d'Alawua, dans le canton de Yongning, pour étudier le moso, son intention initiale était de s’en tenir à une simple analyse phonologique. Il a été surpris de constater que les tons moso n'étaient pas seulement une question purement phonético-phonologique, mais qu'ils étaient aussi inextricablement liés à la grammaire. Cette découverte l'a incité à entreprendre une étude complète du moso. La méthodologie de recherche qu’il déploie s’appuie sur l’important corpus collecté, dont le vocabulaire alimente ce dictionnaire. La compilation du dictionnaire lui-même est un long processus, qui repose sur les méthodes traditionnelles de travail sur le terrain.

La transcription du corpus et son analyse constituent des processus complexes, réalisés de façon systématique. Alexis Michaud considère transcription phonétique et analyse du corpus comme des étapes cruciales et interdépendantes, sur lesquelles repose la validité scientifique et la fiabilité des résultats: pour un dictionnaire, ce qui est en jeu est l'authenticité du lexique. La méthodologie déployée consiste à utiliser l'alphabet phonétique international (API). Récemment, la transcription du parler moso de Alawua bénéficie d’outils logiciels de pointe pour la reconnaissance automatique de la parole, affinés spécifiquement pour le dialecte moso de Alawua. La parole, clairement enregistrée à l'aide d'un équipement audio haute fidélité, est transcrite en texte écrit mot par mot, en préservant le corpus dans sa forme originale, y compris les caractéristiques non linguistiques telles que les répétitions, les pauses et l'intonation.

Dans le processus d'analyse du corpus, Alexis Michaud est attentif à la composition du lexique, à la fréquence d’occurrence des mots, et à la distribution du lexique. Il a analysé les règles grammaticales et les caractéristiques syntaxiques du moso en se basant sur la structure des phrases du corpus. Il a entrepris d’analyser le corpus sous plusieurs angles, à commencer par la phonologie, le vocabulaire et la grammaire, afin de révéler les lois et les caractéristiques intrinsèques de la langue moso.

\section*{Perspectives concernant l’avenir des recherches sur la langue et la culture moso}

Le présent dictionnaire du parler moso du village d'Alawua ne se contente pas d'enregistrer la langue et les traditions culturelles du peuple moso: il reflète également les changements et l'évolution du peuple moso au cours de l'histoire. À travers les explications fournies dans le dictionnaire, on peut déceler la trajectoire évolutive de l'histoire sociale des Moso et les différentes facettes de leur héritage culturel. Ce dictionnaire d’un dialecte moso porte témoignage de ce qui fait l'essence et l'âme de la culture moso. La publication du \emph{Dictionnaire du dialecte moso du village d'Alawua, na – chinois – anglais – français} ouvre un accès à ces précieuses richesses spirituelles. Elles pourront être transmises aux générations futures, qu’elles accompagneront à mesure qu’elles feront leur chemin.

Le processus de compilation du dictionnaire a en outre pour effet, par lui-même, d’amener une forme de normalisation de la langue moso, ce qui contribue à promouvoir le développement des recherches en linguistique, ethnologie et histoire tout en favorisant le dialogue entre disciplines. Enfin, les explications des mots et les phrases-exemples du dictionnaire constituent également des références de qualité pour les apprenants.

Pour conclure, le \emph{Dictionnaire du dialecte moso du village d'Alawua, na – chinois – anglais – français} est une réalisation de bonne tenue, qui favorise une meilleure compréhension de la langue et la culture du peuple moso de la part des universitaires, et fournit une base solide pour des recherches approfondies dans les diverses disciplines concernées. Il est permis d’espérer que ce dictionnaire contribue à améliorer la bonne compréhension et la communication entre les différents groupes ethniques, pour le plus grand bénéfice de l'unité nationale et de l'harmonie sociale. Ce dictionnaire d’un dialecte moso peut jouer un rôle important dans le domaine de l'héritage culturel, et devenir une ressource pédagogique importante dans les régions où vit le peuple moso. L’ouvrage contribue à renforcer les connaissances scientifiques et culturelles du peuple moso, et peut susciter une légitime fierté parmi les locutrices et locuteurs. La nation chinoise est une grande famille composée de 56 nationalités, dont chacune a une langue, une culture et une tradition historique qui lui sont propres. La publication du dictionnaire dialectal moso apporte, à l’édifice des savoirs au sujet des langues et cultures du monde, une pierre précieuse.


{\raggedleft Latami Wangyong \pcmn{拉他咪王勇}\par}

{\raggedleft novembre 2024\par}

%{\raggedleft 2024年初冬时节\par}
